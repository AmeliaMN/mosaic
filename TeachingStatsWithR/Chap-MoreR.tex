\chapter{More About R}








\begin{center}
\it
This material is more advanced than students in Intro Stats need, 

\vspace{-4mm}

but is good for more advanced students and instructors to know.
\end{center}

\DefineShortVerb{\&}

\section{Installing and Using Packages}

\R\ is open source software.  Its development is supported by
a team of core developers and a large community of users.
One way that users support
\R\ is by providing \term{packages} that contain data and functions
for a wide variety of tasks.  

\subsection{Installing packages from \cran}
If you need to install a package, most likely it will be on \cran.
Before a package can be used, it must be \term{installed} 
(once per computer) and 
\term{loaded} (once per \R\ session).  For example, to use \pkg{mosaic}:
\Rindex{install.packages()}%
\Rindex{require()}%

\begin{knitrout}
\definecolor{shadecolor}{rgb}{.97, .97, .97}{\color{fgcolor}\begin{kframe}
\begin{flushleft}
\ttfamily\noindent
\hlfunctioncall{install.packages}\hlkeyword{(}\hlstring{"{}mosaic"{}}\hlkeyword{)}{\ }{\ }\hlcomment{\usebox{\hlnormalsizeboxhash}{\ }fetch{\ }package{\ }from{\ }CRAN{\ }to{\ }local{\ }machine.}\hspace*{\fill}\\
\hlstd{}\hlfunctioncall{require}\hlkeyword{(}\hlsymbol{mosaic}\hlkeyword{)}{\ }{\ }\hlcomment{\usebox{\hlnormalsizeboxhash}{\ }load{\ }the{\ }package{\ }so{\ }it{\ }can{\ }be{\ }used.}\mbox{}
\normalfont
\end{flushleft}
\end{kframe}}
\end{knitrout}


If you are running on a machine where you don't have privileges to
write to the default library location, you can install a personal 
copy of a package.  If the location of your personal library is 
first in &R_LIBS&, this will probably happen automatically.  If not,
you can specify the location manually:

\begin{knitrout}
\definecolor{shadecolor}{rgb}{.97, .97, .97}{\color{fgcolor}\begin{kframe}
\begin{flushleft}
\ttfamily\noindent
\hlfunctioncall{install.packages}\hlkeyword{(}\hlstring{"{}mosaic"{}}\hlkeyword{,}{\ }\hlargument{lib}{\ }\hlargument{=}{\ }\hlstring{"{}\urltilda{}/R/library"{}}\hlkeyword{)}\mbox{}
\normalfont
\end{flushleft}
\end{kframe}}
\end{knitrout}

On a networked machine, be sure to use a different local directory for 
each platform since packages must match the platform.

Installing  packages on a Mac or PC is something you might like to do 
from the GUI since it will provide you with a list of packages from 
which you can select the ones of interest.
Binary packages have been precompiled for a 
particular platform and are generally faster and easier to set up, if they 
are available.  Source packages need to be compiled and built on your local
machine.  Usually this happens automatically -- provided you have all the 
necessary tools installed on your machine -- so the only disadvantage is the
extra time it takes to do the compiling and building.
%Most packages can be obtained as binary packages for PC's and Mac's.

\subsection{Installing other packages}

Occasionally you might find a package of interest that is not available via
a repository like \cran.  
Typically, if you find such a package, you will also find instructions
on how to install it.  If not, you can usually install directly from the 
zipped up package file.

\begin{knitrout}
\definecolor{shadecolor}{rgb}{.97, .97, .97}{\color{fgcolor}\begin{kframe}
\begin{flushleft}
\ttfamily\noindent
\hlfunctioncall{install.packages}\hlkeyword{(}\hlstring{"{}some-package.tar.gz"{}}\hlkeyword{,}{\ }\hlargument{repos}{\ }\hlargument{=}{\ }NULL\hlkeyword{)}{\ }{\ }\hlcomment{\usebox{\hlnormalsizeboxhash}{\ }use{\ }a{\ }file,{\ }not{\ }a{\ }repository}\mbox{}
\normalfont
\end{flushleft}
\end{kframe}}
\end{knitrout}




\subsection{Finding packages}
There are several ways to find packages
\begin{itemize}
  \item Ask your friends.
  \item Google:  Put `cran' in the search.
  \item Rseek:  \url{http://rseek.org} provides a search engine specifically
  designed to find information about \R.
  \item CRAN task views.

  A number of folks have put together task views that annotate a large
  number of packages and summarize they are good for.  They are 
  organized according to themes.  Here are a few examples
  of available task views:
\authNote{NJH to update}

  \UndefineShortVerb{\&}
  \begin{center}
    \begin{tabular}{lp{0.7\textwidth}}
    Bayesian &    Bayesian Inference \\
%    Cluster &   Cluster Analysis \& Finite Mixture Models \\
    Econometrics &   Computational Econometrics \\
%    Environmetrics &   Analysis of ecological and environmental data \\
    Finance &   Empirical Finance \\
    Genetics &   Statistical Genetics \\
    Graphics &   Graphic Displays,  Dynamic Graphics,
          Graphic Devices, and Visualization \\
%    gR &   gRaphical models in R \\
%    MachineLearning &   Machine Learning \& Statistical Learning \\
    Multivariate &   Multivariate Statistics \\
%    Psychometrics &   Psychometric Models and Methods \\
    SocialSciences &   Statistics for the Social Sciences \\
%    Spatial &   Analysis of Spatial Data \\
    \end{tabular}
  \end{center}
  \DefineShortVerb{\&}

  \item Bioconductor (\url{http://www.bioconductor.org/}) 
  and Omegahat (\url{http://www.omegahat.org/R}) 
are another
  sources of packages (specify the \verb!repos=! option to use these).

  \item 
  \textit{R Journal}  (formerly \textit{R News})
  is available via \cran\ and 
  often has articles about new packages and their capabilities.

  \item Write your own.

  You can write your own packages, and it isn't that hard to do (but 
we won't cover this here).
\end{itemize}

%\subsection{Some Useful Packages}
%Below are a few packages that may be of interest.
%In this
%section we present a few ``\R\ extras'' related to the material we have 
%covered so far.  The material is organized by the packages involved.

%\subsubsection*{\texttt{mosaic}}

%\subsubsection*{\texttt{Hmisc}}

\section{Some Workflow Suggestions}

In short: \emph{Think like a programmer.}  
\begin{itemize}
  \item Use \R\ interactively only to get documentation and for quick one-offs.
  \item Store your code in a file.  %# rather than entering it at the prompt.

\smallskip
  You can execute all the code in a file using 

\begin{knitrout}
\definecolor{shadecolor}{rgb}{.97, .97, .97}{\color{fgcolor}\begin{kframe}
\begin{flushleft}
\ttfamily\noindent
\hlfunctioncall{source}\hlkeyword{(}\hlstring{"{}file.R"{}}\hlkeyword{)}\mbox{}
\normalfont
\end{flushleft}
\end{kframe}}
\end{knitrout}

\Rstudio\ has options for executing some or all lines in a file, too. 
See the buttons in the panel for any \R\ script.  (You can create a new \R\ script
file in the main file menu.)

\smallskip\noindent
If you work at the interactive prompt in the console and later wish you had 
been putting your commands into a file, you can save your past commands with

\begin{knitrout}
\definecolor{shadecolor}{rgb}{.97, .97, .97}{\color{fgcolor}\begin{kframe}
\begin{flushleft}
\ttfamily\noindent
\hlfunctioncall{savehistory}\hlkeyword{(}\hlstring{"{}someRCommandsIalmostLost.R"{}}\hlkeyword{)}\mbox{}
\normalfont
\end{flushleft}
\end{kframe}}
\end{knitrout}


You can selectively save portions of your history to a script file
using the History panel in \Rstudio.

\noindent
Then you can go back and edit the file.
\FoodForThought{\RStudio\ makes this especially easy by providing a 
integrated environment for working with \R\ script files, an active
\R\ session, the \R\ session history, etc.  Many of these tasks can 
be done with the click of a button in \RStudio.}



  \item Use meaningful names.
  \item Write reusable functions.

  Learning to write your own functions will greatly increase your efficiency.
  (Stay tuned for details.)

  \item Comment your code.

  It's amazing what you can forget.  The comment character in \R\ is &#&.
  \myindex{comment character in R@comment character in {\sf R} (\texttt{\#})}%
  \Rindex{\#}%
\end{itemize}

\section{Working with Data}
\label{sec:MoreR-Data}%

\subsection{Data in {\sf R} packages}

Data sets in the \verb!datasets! package or any other loaded package
are available via the \verb!data()! function.  Usually, the use
of \verb!data()! is unnecessary, however, since \R\ will search
most loaded packages (they must have been created with the 
lazy-load option) for data sets without the explicit use of 
\verb!data()!.  The \verb!data()! function can be used to 
restore data after it has been modified or to control which package
is used when data sets with the same name appear in multiple packages.

\subsection{Loading data from flat files}
\R\ can read data from a number of file formats.  
The two most useful formats are &.csv& 
(comma separated values) and white space delimited.  
Excel and most statistical packages can read and write data in these 
formats, so these formats  make it easy to transfer data
between different software.  
\R\ provides \function{read.csv()} and \function{read.table()} to
\Rindex{read.csv()}%
\Rindex{read.table()}%
\Rindex{read.file()}%
handle these two situations.  They work nearly identically except for their 
default settings:
\function{read.csv()} assumes that the first line of the file contains
the variable names but \function{read.table()} assumes that the data begins on the first
line with no names for the variables,
and \function{read.table()} will ignore lines that begin with `&#&' but \function{read.csv()} will
not.

The default behavior can be overridden for each function, 
and there are a number of options that make it possible to 
read other file formats,
to omit a specified number of lines at the top of the file, etc.
If you are making the file yourself,
always include meaningful names in either file format. 


It is also possible to read data from a file located on the Internet.  
Simply replace the file name with a URL.
The data read below come from \cite{Tufte:2001:Visual}.
\authNoted{Check citation for Tufte.}%

\begin{knitrout}
\definecolor{shadecolor}{rgb}{.97, .97, .97}{\color{fgcolor}\begin{kframe}
\begin{flushleft}
\ttfamily\noindent
\hlcomment{\usebox{\hlnormalsizeboxhash}{\ }need{\ }header=TRUE{\ }because{\ }there{\ }is{\ }a{\ }header{\ }line.}\hspace*{\fill}\\
\hlstd{}\hlcomment{\usebox{\hlnormalsizeboxhash}{\ }could{\ }also{\ }use{\ }read.file(){\ }without{\ }header=TRUE}\hspace*{\fill}\\
\hlstd{}\hlsymbol{traffic}{\ }\hlassignement{\usebox{\hlnormalsizeboxlessthan}-}{\ }\hlfunctioncall{read.table}\hlkeyword{(}\hlstring{"{}http://www.calvin.edu/\urltilda{}rpruim/fastR/trafficTufte.txt"{}}\hlkeyword{,}\hspace*{\fill}\\
\hlstd{}{\ }{\ }{\ }{\ }\hlargument{header}{\ }\hlargument{=}{\ }\hlnumber{TRUE}\hlkeyword{)}\hspace*{\fill}\\
\hlstd{}\hlsymbol{traffic}\mbox{}
\normalfont
\end{flushleft}
\begin{verbatim}
  year cn.deaths   ny   cn   ma   ri
1 1951       265 13.9 13.0 10.2  8.0
2 1952       230 13.8 10.8 10.0  8.5
3 1953       275 14.4 12.8 11.0  8.5
4 1954       240 13.0 10.8 10.5  7.5
5 1955       325 13.5 14.0 11.8 10.0
6 1956       280 13.4 12.1 11.0  8.2
7 1957       273 13.3 11.9 10.2  9.4
8 1958       248 13.0 10.1 11.8  8.6
9 1959       245 12.9 10.0 11.0  9.0
\end{verbatim}
\end{kframe}}
\end{knitrout}


Notice the use of \code{<-} in the example above.  
This is the \rterm{assignment operator}
%\myindex{<-@\texttt{<-}|seeonly{assignment operator}}%
%\myindex{assignment operator}%
in \R.  
It can be used in either direction (\code{<-} or \code{->}).  In the first line of the 
example above, the results of \function{read.table()} are stored in a variable called 
\dfn{traffic}.  \dfn{traffic} is a \rterm{data frame}, \R's preferred container for
data.  (More about data types in \R\ as we go along.)



The \option{na.strings} argument can be used to specify
codes for missing values.  
The following can be useful for SAS output, for example:
\begin{knitrout}
\definecolor{shadecolor}{rgb}{.97, .97, .97}{\color{fgcolor}\begin{kframe}
\begin{flushleft}
\ttfamily\noindent
\hlsymbol{someData}{\ }\hlassignement{\usebox{\hlnormalsizeboxlessthan}-}{\ }\hlfunctioncall{read.csv}\hlkeyword{(}\hlstring{"{}file.csv"{}}\hlkeyword{,}{\ }\hlargument{na.strings}{\ }\hlargument{=}{\ }\hlfunctioncall{c}\hlkeyword{(}\hlstring{"{}NA"{}}\hlkeyword{,}{\ }\hlstring{"{}"{}}\hlkeyword{,}{\ }\hlstring{"{}."{}}\hlkeyword{,}{\ }\hlstring{"{}na"{}}\hlkeyword{)}\hlkeyword{)}\mbox{}
\normalfont
\end{flushleft}
\end{kframe}}
\end{knitrout}

because SAS uses a period (\verb!.!) to code missing data, but \R\ by default
reads that as string data, which forces the entire variable to be of character 
type instead of numeric.

For convenience the \verb!mosaic! package provides \function{read.file()} which
uses the file name to determine which of \function{read.csv()},
\function{read.table()}, and \function{load()} to use and sets the defaults
to 
\begin{itemize}
\item
\option{header=TRUE}, 
\item
\verb!comment.char="#"!, and 
\item
\verb!na.strings=c('NA','','.','na')! 
\end{itemize}
for \function{read.csv()} and \function{read.table()}.

\begin{knitrout}
\definecolor{shadecolor}{rgb}{.97, .97, .97}{\color{fgcolor}\begin{kframe}
\begin{flushleft}
\ttfamily\noindent
\hlsymbol{traffic}{\ }\hlassignement{\usebox{\hlnormalsizeboxlessthan}-}{\ }\hlfunctioncall{read.file}\hlkeyword{(}\hlstring{"{}http://www.calvin.edu/\urltilda{}rpruim/fastR/trafficTufte.txt"{}}\hlkeyword{)}\mbox{}
\normalfont
\end{flushleft}
\end{kframe}}
\end{knitrout}


\subsection{Manually typing in data}

If you need to enter a small data set by hand, 
the \function{scan()} function is quick and easy.
\Rindex{scan()}%
Individual values are separated by white space or new lines.  
A blank line is used to signal the end of the data.
By default, \function{scan()} is expecting decimal data (which it calls \rterm{double}, 
for double precision), but it is possible to tell \function{scan()} to expect something else,
like \rterm{character} data (i.e., text). 
There are other options for data types, but numerical and text data will usually suffice
for our purposes.  See \code{?scan} for more information and examples.

\begin{Rcode}
myData1 <- scan()
15 18
12
21 23 50 15

myData1

myData2 <- scan(what="character")
"red" "red" "orange" "green" "blue" "blue" "red"

myData2
\end{Rcode}



%
Be sure when using \function{scan()} that you remember to save your data somewhere.
Otherwise you will have to type it again.

\subsection{Creating data frames from vectors}

\Rindex{data.frame()}%
The \function{scan()} function puts data into a \rterm{vector}, not a \rterm{data frame}.  We can
build a \rterm{data frame} for our data as follows.

\begin{knitrout}
\definecolor{shadecolor}{rgb}{.97, .97, .97}{\color{fgcolor}\begin{kframe}
\begin{flushleft}
\ttfamily\noindent
\hlsymbol{myDataFrame}{\ }\hlassignement{\usebox{\hlnormalsizeboxlessthan}-}{\ }\hlfunctioncall{data.frame}\hlkeyword{(}\hlargument{color}{\ }\hlargument{=}{\ }\hlsymbol{myData2}\hlkeyword{,}{\ }\hlargument{number}{\ }\hlargument{=}{\ }\hlsymbol{myData1}\hlkeyword{)}\hspace*{\fill}\\
\hlstd{}\hlsymbol{myDataFrame}\mbox{}
\normalfont
\end{flushleft}
\begin{verbatim}
   color number
1    red     15
2    red     18
3 orange     12
4  green     21
5   blue     23
6   blue     50
7    red     15
\end{verbatim}
\end{kframe}}
\end{knitrout}


\subsection{Getting data from mySQL data bases}

The \pkg{RMySQL} package allows direct access to data in MySQL data bases.
This can be convenient when dealing with subsets of very large data sets.
A great example of this is the 12 gigabytes of data from the Airline on-time
performance dataset included in the 2009 Data Expo (\url{http://stat-computing.org/dataexpo/2009}).
\Rindex{RMySQL}%
\myindex{SQL}%
There is an \href{http://csg.sph.umich.edu/docs/R/rsql.html}{online document}
describing this type of manipulation. 


\subsection{Generating data}
\label{sec:generatingData}
\Rindex{rep()}%
\Rindex{seq()}%
\Rindex{c()}%
\Rindex{rnorm()}%
\Rindex{sample()}%

The following code shows a number of ways to generate data systematically.

\begin{knitrout}
\definecolor{shadecolor}{rgb}{.97, .97, .97}{\color{fgcolor}\begin{kframe}
\begin{flushleft}
\ttfamily\noindent
\hlsymbol{x}{\ }\hlassignement{\usebox{\hlnormalsizeboxlessthan}-}{\ }\hlnumber{5}\hlkeyword{:}\hlnumber{20}\hspace*{\fill}\\
\hlstd{}\hlsymbol{x}{\ }{\ }\hlcomment{\usebox{\hlnormalsizeboxhash}{\ }all{\ }integers{\ }in{\ }a{\ }range}\mbox{}
\normalfont
\end{flushleft}
\begin{verbatim}
 [1]  5  6  7  8  9 10 11 12 13 14 15 16 17 18 19 20
\end{verbatim}
\begin{flushleft}
\ttfamily\noindent
\hlcomment{\usebox{\hlnormalsizeboxhash}{\ }structured{\ }sequences}\hspace*{\fill}\\
\hlstd{}\hlfunctioncall{seq}\hlkeyword{(}\hlnumber{0}\hlkeyword{,}{\ }\hlnumber{50}\hlkeyword{,}{\ }\hlargument{by}{\ }\hlargument{=}{\ }\hlnumber{5}\hlkeyword{)}\mbox{}
\normalfont
\end{flushleft}
\begin{verbatim}
 [1]  0  5 10 15 20 25 30 35 40 45 50
\end{verbatim}
\begin{flushleft}
\ttfamily\noindent
\hlfunctioncall{seq}\hlkeyword{(}\hlnumber{0}\hlkeyword{,}{\ }\hlnumber{50}\hlkeyword{,}{\ }\hlargument{length}{\ }\hlargument{=}{\ }\hlnumber{7}\hlkeyword{)}\mbox{}
\normalfont
\end{flushleft}
\begin{verbatim}
[1]  0.000  8.333 16.667 25.000 33.333 41.667 50.000
\end{verbatim}
\begin{flushleft}
\ttfamily\noindent
\hlfunctioncall{rep}\hlkeyword{(}\hlnumber{1}\hlkeyword{:}\hlnumber{5}\hlkeyword{,}{\ }\hlargument{each}{\ }\hlargument{=}{\ }\hlnumber{3}\hlkeyword{)}\mbox{}
\normalfont
\end{flushleft}
\begin{verbatim}
 [1] 1 1 1 2 2 2 3 3 3 4 4 4 5 5 5
\end{verbatim}
\begin{flushleft}
\ttfamily\noindent
\hlfunctioncall{rep}\hlkeyword{(}\hlnumber{1}\hlkeyword{:}\hlnumber{5}\hlkeyword{,}{\ }\hlargument{times}{\ }\hlargument{=}{\ }\hlnumber{3}\hlkeyword{)}\mbox{}
\normalfont
\end{flushleft}
\begin{verbatim}
 [1] 1 2 3 4 5 1 2 3 4 5 1 2 3 4 5
\end{verbatim}
\begin{flushleft}
\ttfamily\noindent
\hlfunctioncall{c}\hlkeyword{(}\hlnumber{1}\hlkeyword{:}\hlnumber{5}\hlkeyword{,}{\ }\hlnumber{10}\hlkeyword{,}{\ }\hlnumber{3}\hlkeyword{:}\hlnumber{5}\hlkeyword{)}{\ }{\ }\hlcomment{\usebox{\hlnormalsizeboxhash}{\ }c(){\ }concatenates{\ }vectors}\mbox{}
\normalfont
\end{flushleft}
\begin{verbatim}
[1]  1  2  3  4  5 10  3  4  5
\end{verbatim}
\end{kframe}}
\end{knitrout}


\R\ can also sample from several different distributions.

\begin{knitrout}
\definecolor{shadecolor}{rgb}{.97, .97, .97}{\color{fgcolor}\begin{kframe}
\begin{flushleft}
\ttfamily\noindent
\hlfunctioncall{rnorm}\hlkeyword{(}\hlnumber{10}\hlkeyword{,}{\ }\hlargument{mean}{\ }\hlargument{=}{\ }\hlnumber{10}\hlkeyword{,}{\ }\hlargument{sd}{\ }\hlargument{=}{\ }\hlnumber{2}\hlkeyword{)}{\ }{\ }\hlcomment{\usebox{\hlnormalsizeboxhash}{\ }random{\ }draws{\ }from{\ }normal{\ }distribution}\mbox{}
\normalfont
\end{flushleft}
\begin{verbatim}
 [1]  9.592 12.791 10.913  8.458 12.701  8.198  8.215 13.055  8.773 15.302
\end{verbatim}
\begin{flushleft}
\ttfamily\noindent
\hlsymbol{x}{\ }\hlassignement{\usebox{\hlnormalsizeboxlessthan}-}{\ }\hlnumber{5}\hlkeyword{:}\hlnumber{20}{\ }{\ }\hlcomment{\usebox{\hlnormalsizeboxhash}{\ }all{\ }integers{\ }in{\ }a{\ }range}\hspace*{\fill}\\
\hlstd{}\hlfunctioncall{sample}\hlkeyword{(}\hlsymbol{x}\hlkeyword{,}{\ }\hlargument{size}{\ }\hlargument{=}{\ }\hlnumber{5}\hlkeyword{)}{\ }{\ }\hlcomment{\usebox{\hlnormalsizeboxhash}{\ }random{\ }sample{\ }of{\ }size{\ }5{\ }from{\ }x{\ }(no{\ }replacement)}\mbox{}
\normalfont
\end{flushleft}
\begin{verbatim}
[1] 12 10 13 18 14
\end{verbatim}
\end{kframe}}
\end{knitrout}


Functions for sampling from other distributions include
\function{rbinom()},
\function{rchisq()},
\function{rt()},
\function{rf()},
\function{rhyper()},
etc.
See Section~\ref{sec:DiscreteDistributions} for more information.



\subsection{Saving Data}
\function{write.table()} and \function{write.csv()} can be used to save data from \R\ into
delimited flat files.

\begin{knitrout}
\definecolor{shadecolor}{rgb}{.97, .97, .97}{\color{fgcolor}\begin{kframe}
\begin{flushleft}
\ttfamily\noindent
\hlsymbol{ddd}{\ }\hlassignement{\usebox{\hlnormalsizeboxlessthan}-}{\ }\hlfunctioncall{data.frame}\hlkeyword{(}\hlargument{number}{\ }\hlargument{=}{\ }\hlnumber{1}\hlkeyword{:}\hlnumber{5}\hlkeyword{,}{\ }\hlargument{letter}{\ }\hlargument{=}{\ }\hlsymbol{letters}\hlkeyword{[}\hlnumber{1}\hlkeyword{:}\hlnumber{5}\hlkeyword{]}\hlkeyword{)}\hspace*{\fill}\\
\hlstd{}\hlfunctioncall{args}\hlkeyword{(}\hlsymbol{write.table}\hlkeyword{)}\mbox{}
\normalfont
\end{flushleft}
\begin{verbatim}
function (x, file = "", append = FALSE, quote = TRUE, sep = " ", 
    eol = "\n", na = "NA", dec = ".", row.names = TRUE, col.names = TRUE, 
    qmethod = c("escape", "double"), fileEncoding = "") 
NULL
\end{verbatim}
\begin{flushleft}
\ttfamily\noindent
\hlfunctioncall{write.table}\hlkeyword{(}\hlsymbol{ddd}\hlkeyword{,}{\ }\hlstring{"{}ddd.txt"{}}\hlkeyword{)}\hspace*{\fill}\\
\hlstd{}\hlfunctioncall{write.csv}\hlkeyword{(}\hlsymbol{ddd}\hlkeyword{,}{\ }\hlstring{"{}ddd.csv"{}}\hlkeyword{)}\hspace*{\fill}\\
\hlstd{}\hlcomment{\usebox{\hlnormalsizeboxhash}{\ }this{\ }system{\ }call{\ }should{\ }work{\ }on{\ }a{\ }Mac{\ }or{\ }Linux{\ }machine}\hspace*{\fill}\\
\hlstd{}\hlfunctioncall{system}\hlkeyword{(}\hlstring{"{}head{\ }-20{\ }ddd.txt{\ }ddd.csv"{}}\hlkeyword{)}\mbox{}
\normalfont
\end{flushleft}
\end{kframe}}
\end{knitrout}




Data can also be saved in native \R\ format.  Saving data sets 
(and other \R\ objects) using \function{save()} has some advantages over other file formats:
\begin{itemize}
  \item 
  Complete information about the objects is saved, including attributes.
  \item
  Data saved this way takes less space and loads much more quickly.
  \item
  Multiple objects can be saved to and loaded from a single file.
\end{itemize}
The downside is that these files are only readable in \R.

\begin{knitrout}
\definecolor{shadecolor}{rgb}{.97, .97, .97}{\color{fgcolor}\begin{kframe}
\begin{flushleft}
\ttfamily\noindent
\hlsymbol{abc}{\ }\hlassignement{\usebox{\hlnormalsizeboxlessthan}-}{\ }\hlstring{"{}abc"{}}\hspace*{\fill}\\
\hlstd{}\hlsymbol{ddd}{\ }\hlassignement{\usebox{\hlnormalsizeboxlessthan}-}{\ }\hlfunctioncall{data.frame}\hlkeyword{(}\hlargument{number}{\ }\hlargument{=}{\ }\hlnumber{1}\hlkeyword{:}\hlnumber{5}\hlkeyword{,}{\ }\hlargument{letter}{\ }\hlargument{=}{\ }\hlsymbol{letters}\hlkeyword{[}\hlnumber{1}\hlkeyword{:}\hlnumber{5}\hlkeyword{]}\hlkeyword{)}\hspace*{\fill}\\
\hlstd{}\hlfunctioncall{save}\hlkeyword{(}\hlsymbol{ddd}\hlkeyword{,}{\ }\hlsymbol{abc}\hlkeyword{,}{\ }\hlargument{file}{\ }\hlargument{=}{\ }\hlstring{"{}ddd.rda"{}}\hlkeyword{)}{\ }{\ }\hlcomment{\usebox{\hlnormalsizeboxhash}{\ }saves{\ }both{\ }objects{\ }in{\ }a{\ }single{\ }file}\hspace*{\fill}\\
\hlstd{}\hlfunctioncall{load}\hlkeyword{(}\hlstring{"{}ddd.rda"{}}\hlkeyword{)}{\ }{\ }\hlcomment{\usebox{\hlnormalsizeboxhash}{\ }loads{\ }them{\ }both}\mbox{}
\normalfont
\end{flushleft}
\end{kframe}}
\end{knitrout}


For more on importing and exporting data, especially from other
formats, see the 
%\href{http://cran.r-project.org/manuals.html}%
\textit{R Data Import/Export} manual available on \cran.

%\subsection{Making Data Available to Students}

\authNote{Need to decide what to say about making data available to 
students.  Perhaps multiple methods? --RJP}%

\section{Primary \R\ Data Structures}
\label{sec:datastruct}

\subsection{Modes and other attributes} %factors, numeric, character, etc.}
In \R, data is stored in objects.  Each \rterm{object} 
has a \emph{name}, \emph{contents}, and also various \emph{attributes}.
Attributes are used to tell \R\ something about the kind
of data stored in an object and to store other auxiliary information.  
Two important attributes shared 
by all objects are \rterm{mode} and \rterm{length}.

\Rindex{mode()}%
\Rindex{attributes()}%
\Rindex{length()}%
\Rindex{attr()}%
\Rindex{[ ]}%

\begin{knitrout}
\definecolor{shadecolor}{rgb}{.97, .97, .97}{\color{fgcolor}\begin{kframe}
\begin{flushleft}
\ttfamily\noindent
\hlsymbol{w}{\ }\hlassignement{\usebox{\hlnormalsizeboxlessthan}-}{\ }\hlnumber{2.5}\hspace*{\fill}\\
\hlstd{}\hlsymbol{x}{\ }\hlassignement{\usebox{\hlnormalsizeboxlessthan}-}{\ }\hlfunctioncall{c}\hlkeyword{(}\hlnumber{1}\hlkeyword{,}{\ }\hlnumber{2}\hlkeyword{)}\hspace*{\fill}\\
\hlstd{}\hlsymbol{y}{\ }\hlassignement{\usebox{\hlnormalsizeboxlessthan}-}{\ }\hlstring{"{}foo"{}}\hspace*{\fill}\\
\hlstd{}\hlsymbol{z}{\ }\hlassignement{\usebox{\hlnormalsizeboxlessthan}-}{\ }\hlnumber{TRUE}\hspace*{\fill}\\
\hlstd{}\hlsymbol{abc}{\ }\hlassignement{\usebox{\hlnormalsizeboxlessthan}-}{\ }\hlsymbol{letters}\hlkeyword{[}\hlnumber{1}\hlkeyword{:}\hlnumber{3}\hlkeyword{]}\mbox{}
\normalfont
\end{flushleft}
\end{kframe}}
\end{knitrout}


\begin{knitrout}
\definecolor{shadecolor}{rgb}{.97, .97, .97}{\color{fgcolor}\begin{kframe}
\begin{flushleft}
\ttfamily\noindent
\hlfunctioncall{mode}\hlkeyword{(}\hlsymbol{w}\hlkeyword{)}\mbox{}
\normalfont
\end{flushleft}
\begin{verbatim}
[1] "numeric"
\end{verbatim}
\begin{flushleft}
\ttfamily\noindent
\hlfunctioncall{length}\hlkeyword{(}\hlsymbol{w}\hlkeyword{)}\mbox{}
\normalfont
\end{flushleft}
\begin{verbatim}
[1] 1
\end{verbatim}
\begin{flushleft}
\ttfamily\noindent
\hlfunctioncall{mode}\hlkeyword{(}\hlsymbol{x}\hlkeyword{)}\mbox{}
\normalfont
\end{flushleft}
\begin{verbatim}
[1] "numeric"
\end{verbatim}
\begin{flushleft}
\ttfamily\noindent
\hlfunctioncall{length}\hlkeyword{(}\hlsymbol{x}\hlkeyword{)}\mbox{}
\normalfont
\end{flushleft}
\begin{verbatim}
[1] 2
\end{verbatim}
\begin{flushleft}
\ttfamily\noindent
\hlfunctioncall{mode}\hlkeyword{(}\hlsymbol{y}\hlkeyword{)}\mbox{}
\normalfont
\end{flushleft}
\begin{verbatim}
[1] "character"
\end{verbatim}
\begin{flushleft}
\ttfamily\noindent
\hlfunctioncall{length}\hlkeyword{(}\hlsymbol{y}\hlkeyword{)}\mbox{}
\normalfont
\end{flushleft}
\begin{verbatim}
[1] 1
\end{verbatim}
\end{kframe}}
\end{knitrout}


\begin{knitrout}
\definecolor{shadecolor}{rgb}{.97, .97, .97}{\color{fgcolor}\begin{kframe}
\begin{flushleft}
\ttfamily\noindent
\hlsymbol{y}\hlkeyword{[}\hlnumber{1}\hlkeyword{]}\mbox{}
\normalfont
\end{flushleft}
\begin{verbatim}
[1] "foo"
\end{verbatim}
\begin{flushleft}
\ttfamily\noindent
\hlsymbol{y}\hlkeyword{[}\hlnumber{2}\hlkeyword{]}{\ }{\ }\hlcomment{\usebox{\hlnormalsizeboxhash}{\ }not{\ }an{\ }error{\ }to{\ }ask{\ }for{\ }y[2]}\mbox{}
\normalfont
\end{flushleft}
\begin{verbatim}
[1] NA
\end{verbatim}
\begin{flushleft}
\ttfamily\noindent
\hlfunctioncall{mode}\hlkeyword{(}\hlsymbol{z}\hlkeyword{)}\mbox{}
\normalfont
\end{flushleft}
\begin{verbatim}
[1] "logical"
\end{verbatim}
\begin{flushleft}
\ttfamily\noindent
\hlfunctioncall{length}\hlkeyword{(}\hlsymbol{z}\hlkeyword{)}\mbox{}
\normalfont
\end{flushleft}
\begin{verbatim}
[1] 1
\end{verbatim}
\begin{flushleft}
\ttfamily\noindent
\hlsymbol{abc}\mbox{}
\normalfont
\end{flushleft}
\begin{verbatim}
[1] "a" "b" "c"
\end{verbatim}
\begin{flushleft}
\ttfamily\noindent
\hlfunctioncall{mode}\hlkeyword{(}\hlsymbol{abc}\hlkeyword{)}\mbox{}
\normalfont
\end{flushleft}
\begin{verbatim}
[1] "character"
\end{verbatim}
\begin{flushleft}
\ttfamily\noindent
\hlfunctioncall{length}\hlkeyword{(}\hlsymbol{abc}\hlkeyword{)}\mbox{}
\normalfont
\end{flushleft}
\begin{verbatim}
[1] 3
\end{verbatim}
\begin{flushleft}
\ttfamily\noindent
\hlsymbol{abc}\hlkeyword{[}\hlnumber{3}\hlkeyword{]}\mbox{}
\normalfont
\end{flushleft}
\begin{verbatim}
[1] "c"
\end{verbatim}
\end{kframe}}
\end{knitrout}


Each of the objects in the example above is a \rterm{vector}, an ordered container
of values that all have the same mode.%
\footnote{
There are other modes in addition to the ones shown here, including
\code{complex} (for complex numbers), 
\code{function}, \code{list}, \code{call}, and \code{expression}.}
The \function{c()} function concatenates vectors (or lists).
Notice that \code{w}, \code{y}, and \code{z} are 
vectors of length~1.  Missing values are coded as \code{NA} (not available).  Asking
for an entry ``off the end'' of a vector returns \code{NA}.
Assigning a value ``off the end'' of a vector results in the vector being
lengthened so that the new value can be stored in the appropriate location.

There are important ways that \R\ has 
been optimized to work with vectors since they correspond to variables 
(in the sense of statistics).
For categorical data, a \rterm{factor} is a special type of vector that includes
an additional attribute called \emph{levels}.  
A factor can be ordered or unordered (which can affect how statistics
are done and graphs are made) and its elements can have mode
\code{numeric} or \code{character}.

A \rterm{list} is similar to a vector, but its elements may be of different 
modes (including \code{list}, \code{vector}, etc.).
A \rterm{data frame} is a list of vectors (or factors), 
each of the same length, but not necessarily of the same mode.  
This is \R's primary way of storing data sets.
An \rterm{array} is a multi-dimensional table of values that all have the same 
mode.  A \rterm{matrix} is a 2-dimensional array.
\Rindex{matrix()}%

\Rindex{[ ]}%
\Rindex{[[ ]]}%
The access operators (\code{[ ]} for vectors, matrices, arrays, and data frames,
and  \code{[[ ]]} for lists) are actually \emph{functions} in \R.
This has some important consequences:
\begin{itemize}
  \item Accessing elements is slower than in a language like C/C++
  where access is done by pointer arithmetic.
  \item
  These functions also have named arguments, so you can see code like the following
\end{itemize}

\begin{knitrout}
\definecolor{shadecolor}{rgb}{.97, .97, .97}{\color{fgcolor}\begin{kframe}
\begin{flushleft}
\ttfamily\noindent
\hlsymbol{xm}{\ }\hlassignement{\usebox{\hlnormalsizeboxlessthan}-}{\ }\hlfunctioncall{matrix}\hlkeyword{(}\hlnumber{1}\hlkeyword{:}\hlnumber{16}\hlkeyword{,}{\ }\hlargument{nrow}{\ }\hlargument{=}{\ }\hlnumber{4}\hlkeyword{)}\hspace*{\fill}\\
\hlstd{}\hlsymbol{xm}\mbox{}
\normalfont
\end{flushleft}
\begin{verbatim}
     [,1] [,2] [,3] [,4]
[1,]    1    5    9   13
[2,]    2    6   10   14
[3,]    3    7   11   15
[4,]    4    8   12   16
\end{verbatim}
\begin{flushleft}
\ttfamily\noindent
\hlsymbol{xm}\hlkeyword{[}\hlnumber{5}\hlkeyword{]}\mbox{}
\normalfont
\end{flushleft}
\begin{verbatim}
[1] 5
\end{verbatim}
\begin{flushleft}
\ttfamily\noindent
\hlsymbol{xm}\hlkeyword{[}\hlkeyword{,}{\ }\hlnumber{2}\hlkeyword{]}{\ }{\ }\hlcomment{\usebox{\hlnormalsizeboxhash}{\ }this{\ }is{\ }1{\ }dimensional{\ }(a{\ }vector)}\mbox{}
\normalfont
\end{flushleft}
\begin{verbatim}
[1] 5 6 7 8
\end{verbatim}
\begin{flushleft}
\ttfamily\noindent
\hlsymbol{xm}\hlkeyword{[}\hlkeyword{,}{\ }\hlnumber{2}\hlkeyword{,}{\ }\hlargument{drop}{\ }\hlargument{=}{\ }\hlnumber{FALSE}\hlkeyword{]}{\ }{\ }\hlcomment{\usebox{\hlnormalsizeboxhash}{\ }this{\ }is{\ }2{\ }dimensional{\ }(still{\ }a{\ }matrix)}\mbox{}
\normalfont
\end{flushleft}
\begin{verbatim}
     [,1]
[1,]    5
[2,]    6
[3,]    7
[4,]    8
\end{verbatim}
\end{kframe}}
\end{knitrout}


Many objects have a \rterm{dim attribute} that stores the dimension
of the object.  You can change it to change the shape (or even the number
of dimensions) of a vector, matrix, or array.
You can see all of the non-intrinsic attributes (mode and 
length are intrinsic) using &attributes()&, 
and you can set attributes (including 
new ones you make up) using &attr()&.  Some attributes, like dimension,
have special functions for accessing or setting.
The \verb!dim()! function returns the dimensions of an object
as a vector.  Alternatively the number of rows and columns can be 
obtained using \verb!nrow()! and \verb!ncol()!.

\Rindex{dim()}%
\Rindex{nrow()}%
\Rindex{ncol()}%
\Rindex{letters[]}%
\Rindex{names()}%
\Rindex{row.names()}%
\Rindex{attr()}%
\Rindex{attributes()}%

\begin{knitrout}
\definecolor{shadecolor}{rgb}{.97, .97, .97}{\color{fgcolor}\begin{kframe}
\begin{flushleft}
\ttfamily\noindent
\hlsymbol{ddd}{\ }\hlassignement{\usebox{\hlnormalsizeboxlessthan}-}{\ }\hlfunctioncall{data.frame}\hlkeyword{(}\hlargument{number}{\ }\hlargument{=}{\ }\hlnumber{1}\hlkeyword{:}\hlnumber{5}\hlkeyword{,}{\ }\hlargument{letter}{\ }\hlargument{=}{\ }\hlsymbol{letters}\hlkeyword{[}\hlnumber{1}\hlkeyword{:}\hlnumber{5}\hlkeyword{]}\hlkeyword{)}\hspace*{\fill}\\
\hlstd{}\hlfunctioncall{attributes}\hlkeyword{(}\hlsymbol{ddd}\hlkeyword{)}\mbox{}
\normalfont
\end{flushleft}
\begin{verbatim}
$names
[1] "number" "letter"

$row.names
[1] 1 2 3 4 5

$class
[1] "data.frame"

\end{verbatim}
\end{kframe}}
\end{knitrout}

\begin{knitrout}
\definecolor{shadecolor}{rgb}{.97, .97, .97}{\color{fgcolor}\begin{kframe}
\begin{flushleft}
\ttfamily\noindent
\hlfunctioncall{dim}\hlkeyword{(}\hlsymbol{ddd}\hlkeyword{)}\mbox{}
\normalfont
\end{flushleft}
\begin{verbatim}
[1] 5 2
\end{verbatim}
\begin{flushleft}
\ttfamily\noindent
\hlfunctioncall{nrow}\hlkeyword{(}\hlsymbol{ddd}\hlkeyword{)}\mbox{}
\normalfont
\end{flushleft}
\begin{verbatim}
[1] 5
\end{verbatim}
\begin{flushleft}
\ttfamily\noindent
\hlfunctioncall{ncol}\hlkeyword{(}\hlsymbol{ddd}\hlkeyword{)}\mbox{}
\normalfont
\end{flushleft}
\begin{verbatim}
[1] 2
\end{verbatim}
\end{kframe}}
\end{knitrout}

\begin{knitrout}
\definecolor{shadecolor}{rgb}{.97, .97, .97}{\color{fgcolor}\begin{kframe}
\begin{flushleft}
\ttfamily\noindent
\hlfunctioncall{names}\hlkeyword{(}\hlsymbol{ddd}\hlkeyword{)}\mbox{}
\normalfont
\end{flushleft}
\begin{verbatim}
[1] "number" "letter"
\end{verbatim}
\begin{flushleft}
\ttfamily\noindent
\hlfunctioncall{row.names}\hlkeyword{(}\hlsymbol{ddd}\hlkeyword{)}\mbox{}
\normalfont
\end{flushleft}
\begin{verbatim}
[1] "1" "2" "3" "4" "5"
\end{verbatim}
\end{kframe}}
\end{knitrout}



\subsection{What \texttt{is} it?}

\R\ provides a number of functions for testing the mode or class of an object.

\begin{knitrout}
\definecolor{shadecolor}{rgb}{.97, .97, .97}{\color{fgcolor}\begin{kframe}
\begin{flushleft}
\ttfamily\noindent
\hlfunctioncall{mode}\hlkeyword{(}\hlsymbol{xm}\hlkeyword{)}\mbox{}
\normalfont
\end{flushleft}
\begin{verbatim}
[1] "numeric"
\end{verbatim}
\begin{flushleft}
\ttfamily\noindent
\hlfunctioncall{class}\hlkeyword{(}\hlsymbol{xm}\hlkeyword{)}\mbox{}
\normalfont
\end{flushleft}
\begin{verbatim}
[1] "matrix"
\end{verbatim}
\begin{flushleft}
\ttfamily\noindent
\hlfunctioncall{c}\hlkeyword{(}\hlfunctioncall{is.numeric}\hlkeyword{(}\hlsymbol{xm}\hlkeyword{)}\hlkeyword{,}{\ }\hlfunctioncall{is.character}\hlkeyword{(}\hlsymbol{xm}\hlkeyword{)}\hlkeyword{,}{\ }\hlfunctioncall{is.integer}\hlkeyword{(}\hlsymbol{xm}\hlkeyword{)}\hlkeyword{,}{\ }\hlfunctioncall{is.logical}\hlkeyword{(}\hlsymbol{xm}\hlkeyword{)}\hlkeyword{)}\mbox{}
\normalfont
\end{flushleft}
\begin{verbatim}
[1]  TRUE FALSE  TRUE FALSE
\end{verbatim}
\begin{flushleft}
\ttfamily\noindent
\hlfunctioncall{c}\hlkeyword{(}\hlfunctioncall{is.vector}\hlkeyword{(}\hlsymbol{xm}\hlkeyword{)}\hlkeyword{,}{\ }\hlfunctioncall{is.matrix}\hlkeyword{(}\hlsymbol{xm}\hlkeyword{)}\hlkeyword{,}{\ }\hlfunctioncall{is.array}\hlkeyword{(}\hlsymbol{xm}\hlkeyword{)}\hlkeyword{)}\mbox{}
\normalfont
\end{flushleft}
\begin{verbatim}
[1] FALSE  TRUE  TRUE
\end{verbatim}
\end{kframe}}
\end{knitrout}



\subsection{Changing modes (coercion)}
\Rindex{is.numeric()}%
\Rindex{as.numeric()}%
\Rindex{is.integer()}%
\Rindex{as.integer()}%

If \R\ is expecting an object of a certain mode or class but gets 
something else, it will often try to \rterm{coerce} the object to meet 
its expectations.  You
can also coerce things manually using one of the many &as.???()& functions.

\begin{knitrout}
\definecolor{shadecolor}{rgb}{.97, .97, .97}{\color{fgcolor}\begin{kframe}
\begin{flushleft}
\ttfamily\noindent
\hlfunctioncall{apropos}\hlkeyword{(}\hlstring{"{}\usebox{\hlnormalsizeboxhat}as\usebox{\hlnormalsizeboxbackslash}\usebox{\hlnormalsizeboxbackslash}."{}}\hlkeyword{)}\hlkeyword{[}\hlnumber{1}\hlkeyword{:}\hlnumber{10}\hlkeyword{]}{\ }{\ }\hlcomment{\usebox{\hlnormalsizeboxhash}{\ }just{\ }a{\ }small{\ }sample}\mbox{}
\normalfont
\end{flushleft}
\begin{verbatim}
 [1] "as.array"               "as.array.default"       "as.call"               
 [4] "as.category"            "as.character"           "as.character.condition"
 [7] "as.character.Date"      "as.character.default"   "as.character.error"    
[10] "as.character.factor"   
\end{verbatim}
\begin{flushleft}
\ttfamily\noindent
\hlsymbol{xm}\mbox{}
\normalfont
\end{flushleft}
\begin{verbatim}
     [,1] [,2] [,3] [,4]
[1,]    1    5    9   13
[2,]    2    6   10   14
[3,]    3    7   11   15
[4,]    4    8   12   16
\end{verbatim}
\begin{flushleft}
\ttfamily\noindent
\hlcomment{\usebox{\hlnormalsizeboxhash}{\ }convert{\ }numbers{\ }to{\ }strings{\ }(this{\ }drops{\ }attributes,{\ }including{\ }dimension)}\hspace*{\fill}\\
\hlstd{}\hlfunctioncall{as.character}\hlkeyword{(}\hlsymbol{xm}\hlkeyword{)}\mbox{}
\normalfont
\end{flushleft}
\begin{verbatim}
 [1] "1"  "2"  "3"  "4"  "5"  "6"  "7"  "8"  "9"  "10" "11" "12" "13" "14" "15" "16"
\end{verbatim}
\end{kframe}}
\end{knitrout}

\begin{knitrout}
\definecolor{shadecolor}{rgb}{.97, .97, .97}{\color{fgcolor}\begin{kframe}
\begin{flushleft}
\ttfamily\noindent
\hlcomment{\usebox{\hlnormalsizeboxhash}{\ }convert{\ }matrix{\ }to{\ }vector}\hspace*{\fill}\\
\hlstd{}\hlfunctioncall{as.vector}\hlkeyword{(}\hlsymbol{xm}\hlkeyword{)}\mbox{}
\normalfont
\end{flushleft}
\begin{verbatim}
 [1]  1  2  3  4  5  6  7  8  9 10 11 12 13 14 15 16
\end{verbatim}
\begin{flushleft}
\ttfamily\noindent
\hlfunctioncall{as.logical}\hlkeyword{(}\hlsymbol{xm}\hlkeyword{)}\mbox{}
\normalfont
\end{flushleft}
\begin{verbatim}
 [1] TRUE TRUE TRUE TRUE TRUE TRUE TRUE TRUE TRUE TRUE TRUE TRUE TRUE TRUE TRUE TRUE
\end{verbatim}
\end{kframe}}
\end{knitrout}

\begin{knitrout}
\definecolor{shadecolor}{rgb}{.97, .97, .97}{\color{fgcolor}\begin{kframe}
\begin{flushleft}
\ttfamily\noindent
\hlsymbol{alpha}{\ }\hlassignement{\usebox{\hlnormalsizeboxlessthan}-}{\ }\hlfunctioncall{c}\hlkeyword{(}\hlstring{"{}a"{}}\hlkeyword{,}{\ }\hlstring{"{}1"{}}\hlkeyword{,}{\ }\hlstring{"{}b"{}}\hlkeyword{,}{\ }\hlstring{"{}0.5"{}}\hlkeyword{)}\hspace*{\fill}\\
\hlstd{}\hlfunctioncall{mode}\hlkeyword{(}\hlsymbol{alpha}\hlkeyword{)}\mbox{}
\normalfont
\end{flushleft}
\begin{verbatim}
[1] "character"
\end{verbatim}
\end{kframe}}
\end{knitrout}

\begin{knitrout}
\definecolor{shadecolor}{rgb}{.97, .97, .97}{\color{fgcolor}\begin{kframe}
\begin{flushleft}
\ttfamily\noindent
\hlfunctioncall{as.numeric}\hlkeyword{(}\hlsymbol{alpha}\hlkeyword{)}{\ }{\ }\hlcomment{\usebox{\hlnormalsizeboxhash}{\ }can\usebox{\hlnormalsizeboxsinglequote}t{\ }do{\ }the{\ }coercion,{\ }so{\ }NAs{\ }are{\ }introduced}\mbox{}
\normalfont
\end{flushleft}
\begin{verbatim}
Warning message: NAs introduced by coercion
\end{verbatim}
\begin{verbatim}
[1]  NA 1.0  NA 0.5
\end{verbatim}
\begin{flushleft}
\ttfamily\noindent
\hlfunctioncall{as.integer}\hlkeyword{(}\hlsymbol{alpha}\hlkeyword{)}{\ }{\ }\hlcomment{\usebox{\hlnormalsizeboxhash}{\ }notice{\ }coercion{\ }of{\ }0.5{\ }to{\ }0}\mbox{}
\normalfont
\end{flushleft}
\begin{verbatim}
Warning message: NAs introduced by coercion
\end{verbatim}
\begin{verbatim}
[1] NA  1 NA  0
\end{verbatim}
\end{kframe}}
\end{knitrout}


\section{More About Vectors}
\label{sec:Rvectors}
Vectors are so important in \R\ that they deserve some additional discussion.
In Section~\ref{sec:generatingData} we learned how to generate some simple
vectors.  Here we will learn about some of the operations and functions
that can be applied to vectors.

\subsection{Names and vectors}

We can give each position in a vector a name.  This can be very handy for certain uses
of vectors.

\begin{knitrout}
\definecolor{shadecolor}{rgb}{.97, .97, .97}{\color{fgcolor}\begin{kframe}
\begin{flushleft}
\ttfamily\noindent
\hlsymbol{myvec}{\ }\hlassignement{\usebox{\hlnormalsizeboxlessthan}-}{\ }\hlnumber{1}\hlkeyword{:}\hlnumber{5}\hspace*{\fill}\\
\hlstd{}\hlsymbol{myvec}\mbox{}
\normalfont
\end{flushleft}
\begin{verbatim}
[1] 1 2 3 4 5
\end{verbatim}
\begin{flushleft}
\ttfamily\noindent
\hlfunctioncall{names}\hlkeyword{(}\hlsymbol{myvec}\hlkeyword{)}{\ }\hlassignement{\usebox{\hlnormalsizeboxlessthan}-}{\ }\hlfunctioncall{c}\hlkeyword{(}\hlstring{"{}one"{}}\hlkeyword{,}{\ }\hlstring{"{}two"{}}\hlkeyword{,}{\ }\hlstring{"{}three"{}}\hlkeyword{,}{\ }\hlstring{"{}four"{}}\hlkeyword{,}{\ }\hlstring{"{}five"{}}\hlkeyword{)}\hspace*{\fill}\\
\hlstd{}\hlsymbol{myvec}\mbox{}
\normalfont
\end{flushleft}
\begin{verbatim}
  one   two three  four  five 
    1     2     3     4     5 
\end{verbatim}
\end{kframe}}
\end{knitrout}


Names can also be specified as a vector is created using \function{c()}.
\begin{knitrout}
\definecolor{shadecolor}{rgb}{.97, .97, .97}{\color{fgcolor}\begin{kframe}
\begin{flushleft}
\ttfamily\noindent
\hlsymbol{another}{\ }\hlassignement{\usebox{\hlnormalsizeboxlessthan}-}{\ }\hlfunctioncall{c}\hlkeyword{(}\hlargument{mean}{\ }\hlargument{=}{\ }\hlnumber{10}\hlkeyword{,}{\ }\hlargument{sd}{\ }\hlargument{=}{\ }\hlnumber{2}\hlkeyword{,}{\ }\hlargument{\usebox{\hlnormalsizeboxbacktick}trimmed{\ }mean\usebox{\hlnormalsizeboxbacktick}}{\ }\hlargument{=}{\ }\hlnumber{9.7}\hlkeyword{)}\hspace*{\fill}\\
\hlstd{}\hlsymbol{another}\mbox{}
\normalfont
\end{flushleft}
\begin{verbatim}
        mean           sd trimmed mean 
        10.0          2.0          9.7 
\end{verbatim}
\end{kframe}}
\end{knitrout}

\subsection{Vectorized functions}

Many \R\ functions and operations are ``vectorized'' and can be applied
not just to an individual value but to an entire vector, in which case
they are applied componentwise and return a vector of transformed values.  
Most traditional mathematics functions are available and work this way.
\Rindex{mean()}%
\Rindex{sd()}%
\Rindex{var()}%
\Rindex{median()}%
\Rindex{log()}%

\begin{knitrout}
\definecolor{shadecolor}{rgb}{.97, .97, .97}{\color{fgcolor}\begin{kframe}
\begin{flushleft}
\ttfamily\noindent
\hlsymbol{x}{\ }\hlassignement{\usebox{\hlnormalsizeboxlessthan}-}{\ }\hlnumber{1}\hlkeyword{:}\hlnumber{5}\hspace*{\fill}\\
\hlstd{}\hlsymbol{y}{\ }\hlassignement{\usebox{\hlnormalsizeboxlessthan}-}{\ }\hlfunctioncall{seq}\hlkeyword{(}\hlnumber{10}\hlkeyword{,}{\ }\hlnumber{60}\hlkeyword{,}{\ }\hlargument{by}{\ }\hlargument{=}{\ }\hlnumber{10}\hlkeyword{)}\hspace*{\fill}\\
\hlstd{}\hlsymbol{z}{\ }\hlassignement{\usebox{\hlnormalsizeboxlessthan}-}{\ }\hlfunctioncall{rnorm}\hlkeyword{(}\hlnumber{10}\hlkeyword{)}\hspace*{\fill}\\
\hlstd{}\hlsymbol{x}\mbox{}
\normalfont
\end{flushleft}
\begin{verbatim}
[1] 1 2 3 4 5
\end{verbatim}
\begin{flushleft}
\ttfamily\noindent
\hlsymbol{y}\mbox{}
\normalfont
\end{flushleft}
\begin{verbatim}
[1] 10 20 30 40 50 60
\end{verbatim}
\begin{flushleft}
\ttfamily\noindent
\hlsymbol{y}{\ }\hlkeyword{+}{\ }\hlnumber{1}\mbox{}
\normalfont
\end{flushleft}
\begin{verbatim}
[1] 11 21 31 41 51 61
\end{verbatim}
\begin{flushleft}
\ttfamily\noindent
\hlsymbol{x}{\ }\hlkeyword{*}{\ }\hlnumber{10}\mbox{}
\normalfont
\end{flushleft}
\begin{verbatim}
[1] 10 20 30 40 50
\end{verbatim}
\begin{flushleft}
\ttfamily\noindent
\hlsymbol{x}{\ }\hlkeyword{\usebox{\hlnormalsizeboxlessthan}}{\ }\hlnumber{3}\mbox{}
\normalfont
\end{flushleft}
\begin{verbatim}
[1]  TRUE  TRUE FALSE FALSE FALSE
\end{verbatim}
\begin{flushleft}
\ttfamily\noindent
\hlsymbol{x}\hlkeyword{\usebox{\hlnormalsizeboxhat}}\hlnumber{2}\mbox{}
\normalfont
\end{flushleft}
\begin{verbatim}
[1]  1  4  9 16 25
\end{verbatim}
\begin{flushleft}
\ttfamily\noindent
\hlfunctioncall{log}\hlkeyword{(}\hlsymbol{x}\hlkeyword{)}\mbox{}
\normalfont
\end{flushleft}
\begin{verbatim}
[1] 0.0000 0.6931 1.0986 1.3863 1.6094
\end{verbatim}
\begin{flushleft}
\ttfamily\noindent
\hlfunctioncall{log}\hlkeyword{(}\hlsymbol{x}\hlkeyword{,}{\ }\hlargument{base}{\ }\hlargument{=}{\ }\hlnumber{10}\hlkeyword{)}{\ }{\ }\hlcomment{\usebox{\hlnormalsizeboxhash}{\ }natural{\ }and{\ }base{\ }10{\ }logs}\mbox{}
\normalfont
\end{flushleft}
\begin{verbatim}
[1] 0.0000 0.3010 0.4771 0.6021 0.6990
\end{verbatim}
\end{kframe}}
\end{knitrout}


\noindent
Vectors can be combined into a matrix using &rbind()& or &cbind()&.  
This can facilitate side-by-side comparisons.
\Rindex{rbind()}%
\Rindex{cbind()}%
\Rindex{round()}%
\Rindex{signif()}%

\begin{knitrout}
\definecolor{shadecolor}{rgb}{.97, .97, .97}{\color{fgcolor}\begin{kframe}
\begin{flushleft}
\ttfamily\noindent
\hlcomment{\usebox{\hlnormalsizeboxhash}{\ }compare{\ }round(){\ }and{\ }signif(){\ }by{\ }binding{\ }rowwise{\ }into{\ }matrix}\hspace*{\fill}\\
\hlstd{}\hlfunctioncall{rbind}\hlkeyword{(}\hlfunctioncall{round}\hlkeyword{(}\hlsymbol{z}\hlkeyword{,}{\ }\hlargument{digits}{\ }\hlargument{=}{\ }\hlnumber{2}\hlkeyword{)}\hlkeyword{,}{\ }\hlfunctioncall{signif}\hlkeyword{(}\hlsymbol{z}\hlkeyword{,}{\ }\hlargument{digits}{\ }\hlargument{=}{\ }\hlnumber{2}\hlkeyword{)}\hlkeyword{)}\mbox{}
\normalfont
\end{flushleft}
\begin{verbatim}
       [,1] [,2] [,3]  [,4]  [,5] [,6] [,7] [,8] [,9] [,10]
[1,] -0.050 0.85 -1.3 0.100 -0.43 0.13 0.43 -0.2 -0.6 -0.15
[2,] -0.052 0.85 -1.3 0.097 -0.43 0.13 0.43 -0.2 -0.6 -0.15
\end{verbatim}
\end{kframe}}
\end{knitrout}



\subsection{Functions that act on vectors as vectors}

Other functions, including many statistical functions,
are designed to work on the vector as a vector.  Often these 
return a single value (technically a vector of length~1), but
other return types are used as appropriate.

\begin{knitrout}
\definecolor{shadecolor}{rgb}{.97, .97, .97}{\color{fgcolor}\begin{kframe}
\begin{flushleft}
\ttfamily\noindent
\hlsymbol{x}{\ }\hlassignement{\usebox{\hlnormalsizeboxlessthan}-}{\ }\hlnumber{1}\hlkeyword{:}\hlnumber{10}\hspace*{\fill}\\
\hlstd{}\hlsymbol{z}{\ }\hlassignement{\usebox{\hlnormalsizeboxlessthan}-}{\ }\hlfunctioncall{rnorm}\hlkeyword{(}\hlnumber{100}\hlkeyword{)}\hspace*{\fill}\\
\hlstd{}\hlfunctioncall{mean}\hlkeyword{(}\hlsymbol{z}\hlkeyword{)}\mbox{}
\normalfont
\end{flushleft}
\begin{verbatim}
   mean 
-0.0412 
\end{verbatim}
\begin{flushleft}
\ttfamily\noindent
\hlfunctioncall{sd}\hlkeyword{(}\hlsymbol{z}\hlkeyword{)}\mbox{}
\normalfont
\end{flushleft}
\begin{verbatim}
    sd 
0.9808 
\end{verbatim}
\begin{flushleft}
\ttfamily\noindent
\hlfunctioncall{var}\hlkeyword{(}\hlsymbol{z}\hlkeyword{)}\mbox{}
\normalfont
\end{flushleft}
\begin{verbatim}
   var 
0.9619 
\end{verbatim}
\begin{flushleft}
\ttfamily\noindent
\hlfunctioncall{median}\hlkeyword{(}\hlsymbol{z}\hlkeyword{)}{\ }{\ }\hlcomment{\usebox{\hlnormalsizeboxhash}{\ }basic{\ }statistical{\ }functions}\mbox{}
\normalfont
\end{flushleft}
\begin{verbatim}
 median 
-0.1279 
\end{verbatim}
\begin{flushleft}
\ttfamily\noindent
\hlfunctioncall{range}\hlkeyword{(}\hlsymbol{z}\hlkeyword{)}{\ }{\ }\hlcomment{\usebox{\hlnormalsizeboxhash}{\ }range{\ }returns{\ }a{\ }vector{\ }of{\ }length{\ }2}\mbox{}
\normalfont
\end{flushleft}
\begin{verbatim}
[1] -2.702  2.984
\end{verbatim}
\begin{flushleft}
\ttfamily\noindent
\hlfunctioncall{sum}\hlkeyword{(}\hlsymbol{x}\hlkeyword{)}\mbox{}
\normalfont
\end{flushleft}
\begin{verbatim}
[1] 55
\end{verbatim}
\begin{flushleft}
\ttfamily\noindent
\hlfunctioncall{prod}\hlkeyword{(}\hlsymbol{x}\hlkeyword{)}{\ }{\ }\hlcomment{\usebox{\hlnormalsizeboxhash}{\ }sums{\ }and{\ }products}\mbox{}
\normalfont
\end{flushleft}
\begin{verbatim}
[1] 3628800
\end{verbatim}
\end{kframe}}
\end{knitrout}


\begin{knitrout}
\definecolor{shadecolor}{rgb}{.97, .97, .97}{\color{fgcolor}\begin{kframe}
\begin{flushleft}
\ttfamily\noindent
\hlsymbol{z}{\ }\hlassignement{\usebox{\hlnormalsizeboxlessthan}-}{\ }\hlfunctioncall{rnorm}\hlkeyword{(}\hlnumber{5}\hlkeyword{)}\hspace*{\fill}\\
\hlstd{}\hlsymbol{z}\mbox{}
\normalfont
\end{flushleft}
\begin{verbatim}
[1] -1.9358 -1.2877  0.2551  0.1804  0.4576
\end{verbatim}
\begin{flushleft}
\ttfamily\noindent
\hlfunctioncall{sort}\hlkeyword{(}\hlsymbol{z}\hlkeyword{)}\mbox{}
\normalfont
\end{flushleft}
\begin{verbatim}
[1] -1.9358 -1.2877  0.1804  0.2551  0.4576
\end{verbatim}
\begin{flushleft}
\ttfamily\noindent
\hlfunctioncall{rank}\hlkeyword{(}\hlsymbol{z}\hlkeyword{)}\mbox{}
\normalfont
\end{flushleft}
\begin{verbatim}
[1] 1 2 4 3 5
\end{verbatim}
\begin{flushleft}
\ttfamily\noindent
\hlfunctioncall{order}\hlkeyword{(}\hlsymbol{z}\hlkeyword{)}{\ }{\ }\hlcomment{\usebox{\hlnormalsizeboxhash}{\ }sort,{\ }rank,{\ }order}\mbox{}
\normalfont
\end{flushleft}
\begin{verbatim}
[1] 1 2 4 3 5
\end{verbatim}
\begin{flushleft}
\ttfamily\noindent
\hlfunctioncall{rev}\hlkeyword{(}\hlsymbol{x}\hlkeyword{)}{\ }{\ }\hlcomment{\usebox{\hlnormalsizeboxhash}{\ }reverse{\ }x}\mbox{}
\normalfont
\end{flushleft}
\begin{verbatim}
 [1] 10  9  8  7  6  5  4  3  2  1
\end{verbatim}
\end{kframe}}
\end{knitrout}


\begin{knitrout}
\definecolor{shadecolor}{rgb}{.97, .97, .97}{\color{fgcolor}\begin{kframe}
\begin{flushleft}
\ttfamily\noindent
\hlfunctioncall{diff}\hlkeyword{(}\hlsymbol{x}\hlkeyword{)}{\ }{\ }\hlcomment{\usebox{\hlnormalsizeboxhash}{\ }pairwise{\ }differences}\mbox{}
\normalfont
\end{flushleft}
\begin{verbatim}
[1] 1 1 1 1 1 1 1 1 1
\end{verbatim}
\begin{flushleft}
\ttfamily\noindent
\hlfunctioncall{cumsum}\hlkeyword{(}\hlsymbol{x}\hlkeyword{)}{\ }{\ }\hlcomment{\usebox{\hlnormalsizeboxhash}{\ }cumulative{\ }sum}\mbox{}
\normalfont
\end{flushleft}
\begin{verbatim}
 [1]  1  3  6 10 15 21 28 36 45 55
\end{verbatim}
\begin{flushleft}
\ttfamily\noindent
\hlfunctioncall{cumprod}\hlkeyword{(}\hlsymbol{x}\hlkeyword{)}{\ }{\ }\hlcomment{\usebox{\hlnormalsizeboxhash}{\ }cumulative{\ }product}\mbox{}
\normalfont
\end{flushleft}
\begin{verbatim}
 [1]       1       2       6      24     120     720    5040   40320  362880 3628800
\end{verbatim}
\begin{flushleft}
\ttfamily\noindent
\hlfunctioncall{sum}\hlkeyword{(}\hlsymbol{x}\hlkeyword{)}\mbox{}
\normalfont
\end{flushleft}
\begin{verbatim}
[1] 55
\end{verbatim}
\begin{flushleft}
\ttfamily\noindent
\hlfunctioncall{prod}\hlkeyword{(}\hlsymbol{x}\hlkeyword{)}{\ }{\ }\hlcomment{\usebox{\hlnormalsizeboxhash}{\ }sums{\ }and{\ }products}\mbox{}
\normalfont
\end{flushleft}
\begin{verbatim}
[1] 3628800
\end{verbatim}
\end{kframe}}
\end{knitrout}

\label{r:sumprod}%

Whether a function is vectorized or treats a vector as a unit
depends on its implementation.  Usually, things are implemented 
the way you would expect.  Occasionally you may discover a function
that you wish were vectorized and is not.    
When writing your own functions, give some thought to whether they
should be vectorized, and test them with vectors of length greater than 1
to make sure you get the intended behavior.
\Rindex{sum()}%
\Rindex{prod()}%
\Rindex{cumsum()}%
\Rindex{cumprod()}%
\Rindex{cummin()}%
\Rindex{cummax()}%
\Rindex{diff()}%
\Rindex{rev()}%
\Rindex{sort()}%
\Rindex{rank()}%
\Rindex{order()}%
\Rindex{which()}%
\Rindex{any()}%
\Rindex{unique()}%
\Rindex{table()}%
\Rindex{paste()}%
\Rindex{na.omit()}%
\Rindex{pmin()}%
\Rindex{pmax()}%


Some additional useful functions are included in Table~\ref{table:useful-functions}.

\begin{table}
\caption{Some useful \R\ functions.}
\label{table:useful-functions}%
\begin{center}
\UndefineShortVerb{\&}
  \begin{longtable}{|p{1.2in}|p{3.5in}|}
  \hline
  \verb!cumsum()!

  \verb!cumprod()!

  \verb!cummin()!

  \verb!cummax()!
  &
  Returns vector of cumulative sums, products, minima, or maxima.
  \\ \hline
  \verb!pmin(x,y,...)!

  \verb!pmax(x,y,...)!
  &
  Returns vector of parallel minima or maxima where $i$th element is
  max or min of \verb!x[i]!, \verb!y[i]!, \dots.
  \\ \hline
  \verb!which(x)! 
  &
  Returns a vector of indices of elements of \verb!x! that are true.
  Typical use: \verb!which(y > 5)! returns the indices where elements
  of \verb&y& are larger than 5.
  \\ \hline
  \verb!any(x)! 
  &
  Returns a \verb!logical! indicating whether any elements of \verb!x! 
  are true.
  Typical use: \verb!if ( any(y > 5) ) { ...}!.
  \\ \hline
  \verb!na.omit(x)! & Returns a vector with missing values removed.
  \\ \hline
  \verb!unique(x)! & Returns a vector with repeated values removed.
  \\ \hline
  \verb!table(x)! & Returns a table of counts of the number of 
  occurrences of each value in \verb!x!.  The table is similar
  to a vector with names indicating the values, but it is not a vector.
  \\ \hline
  \verb!paste(x,y,...,!
  
  \verb!  sep=" ")! 
  & Pastes \verb!x! and \verb!y! together
  componentwise (as strings) with \verb!sep! between elements.
  Recycling applies.
  \\ \hline
  \end{longtable}
\DefineShortVerb{\&}
\end{center}
\end{table}


\subsection{Recycling}
\myindex{recycling}%
When vectors operate on each other, the operation is done componentwise, 
recycling the shorter vector to match the length of the longer.

\begin{knitrout}
\definecolor{shadecolor}{rgb}{.97, .97, .97}{\color{fgcolor}\begin{kframe}
\begin{flushleft}
\ttfamily\noindent
\hlsymbol{x}{\ }\hlassignement{\usebox{\hlnormalsizeboxlessthan}-}{\ }\hlnumber{1}\hlkeyword{:}\hlnumber{5}\hspace*{\fill}\\
\hlstd{}\hlsymbol{y}{\ }\hlassignement{\usebox{\hlnormalsizeboxlessthan}-}{\ }\hlfunctioncall{seq}\hlkeyword{(}\hlnumber{10}\hlkeyword{,}{\ }\hlnumber{70}\hlkeyword{,}{\ }\hlargument{by}{\ }\hlargument{=}{\ }\hlnumber{10}\hlkeyword{)}\hspace*{\fill}\\
\hlstd{}\hlsymbol{x}{\ }\hlkeyword{+}{\ }\hlsymbol{y}\mbox{}
\normalfont
\end{flushleft}
\begin{verbatim}
Warning message: longer object length is not a multiple of shorter object length
\end{verbatim}
\begin{verbatim}
[1] 11 22 33 44 55 61 72
\end{verbatim}
\end{kframe}}
\end{knitrout}


\noindent
In fact, this is exactly how things like &x + 1& actually work.
If &x& is a vector of length $n$, then \verb!1! (a vector of length 1) is 
first recycled into a vector of length $n$; then the two vectors are
added componentwise.
Some vectorized functions that take multiple vectors as arguments
will first use recycling to make them the same length.

\subsection{Accessing elements of vectors}
\R\ allows for some very interesting and useful methods for accessing
elements of a vector that combine the ideas above.
First, recall that the &[ ]& operator is actually a function.
Furthermore, it is vectorized.

\begin{knitrout}
\definecolor{shadecolor}{rgb}{.97, .97, .97}{\color{fgcolor}\begin{kframe}
\begin{flushleft}
\ttfamily\noindent
\hlsymbol{x}{\ }\hlassignement{\usebox{\hlnormalsizeboxlessthan}-}{\ }\hlfunctioncall{seq}\hlkeyword{(}\hlnumber{2}\hlkeyword{,}{\ }\hlnumber{20}\hlkeyword{,}{\ }\hlargument{by}{\ }\hlargument{=}{\ }\hlnumber{2}\hlkeyword{)}\hspace*{\fill}\\
\hlstd{}\hlsymbol{x}\hlkeyword{[}\hlnumber{1}\hlkeyword{:}\hlnumber{5}\hlkeyword{]}\mbox{}
\normalfont
\end{flushleft}
\begin{verbatim}
[1]  2  4  6  8 10
\end{verbatim}
\begin{flushleft}
\ttfamily\noindent
\hlsymbol{x}\hlkeyword{[}\hlfunctioncall{c}\hlkeyword{(}\hlnumber{1}\hlkeyword{,}{\ }\hlnumber{4}\hlkeyword{,}{\ }\hlnumber{7}\hlkeyword{)}\hlkeyword{]}\mbox{}
\normalfont
\end{flushleft}
\begin{verbatim}
[1]  2  8 14
\end{verbatim}
\end{kframe}}
\end{knitrout}


&[ ]& accepts &logicals& as arguments well.
The boolean values (recycled, if necessary)
are used to select or deselect elements of the vector.

\begin{knitrout}
\definecolor{shadecolor}{rgb}{.97, .97, .97}{\color{fgcolor}\begin{kframe}
\begin{flushleft}
\ttfamily\noindent
\hlsymbol{x}{\ }\hlassignement{\usebox{\hlnormalsizeboxlessthan}-}{\ }\hlfunctioncall{seq}\hlkeyword{(}\hlnumber{2}\hlkeyword{,}{\ }\hlnumber{20}\hlkeyword{,}{\ }\hlargument{by}{\ }\hlargument{=}{\ }\hlnumber{2}\hlkeyword{)}\hspace*{\fill}\\
\hlstd{}\hlsymbol{x}\hlkeyword{[}\hlfunctioncall{c}\hlkeyword{(}\hlnumber{TRUE}\hlkeyword{,}{\ }\hlnumber{TRUE}\hlkeyword{,}{\ }\hlnumber{FALSE}\hlkeyword{)}\hlkeyword{]}{\ }{\ }\hlcomment{\usebox{\hlnormalsizeboxhash}{\ }skips{\ }every{\ }third{\ }element{\ }(recycling!)}\mbox{}
\normalfont
\end{flushleft}
\begin{verbatim}
[1]  2  4  8 10 14 16 20
\end{verbatim}
\begin{flushleft}
\ttfamily\noindent
\hlsymbol{x}\hlkeyword{[}\hlsymbol{x}{\ }\hlkeyword{\usebox{\hlnormalsizeboxgreaterthan}}{\ }\hlnumber{10}\hlkeyword{]}{\ }{\ }\hlcomment{\usebox{\hlnormalsizeboxhash}{\ }more{\ }typical{\ }use{\ }of{\ }boolean{\ }in{\ }selection}\mbox{}
\normalfont
\end{flushleft}
\begin{verbatim}
[1] 12 14 16 18 20
\end{verbatim}
\end{kframe}}
\end{knitrout}


\noindent
Negative indices are used to omit elements.

\begin{knitrout}
\definecolor{shadecolor}{rgb}{.97, .97, .97}{\color{fgcolor}\begin{kframe}
\begin{flushleft}
\ttfamily\noindent
\hlsymbol{x}{\ }\hlassignement{\usebox{\hlnormalsizeboxlessthan}-}{\ }\hlfunctioncall{seq}\hlkeyword{(}\hlnumber{2}\hlkeyword{,}{\ }\hlnumber{20}\hlkeyword{,}{\ }\hlargument{by}{\ }\hlargument{=}{\ }\hlnumber{2}\hlkeyword{)}\hspace*{\fill}\\
\hlstd{}\hlsymbol{x}\hlkeyword{[}\hlfunctioncall{c}\hlkeyword{(}\hlnumber{TRUE}\hlkeyword{,}{\ }\hlnumber{TRUE}\hlkeyword{,}{\ }\hlnumber{FALSE}\hlkeyword{)}\hlkeyword{]}{\ }{\ }\hlcomment{\usebox{\hlnormalsizeboxhash}{\ }skips{\ }every{\ }third{\ }element{\ }(recycling!)}\mbox{}
\normalfont
\end{flushleft}
\begin{verbatim}
[1]  2  4  8 10 14 16 20
\end{verbatim}
\begin{flushleft}
\ttfamily\noindent
\hlsymbol{x}\hlkeyword{[}\hlsymbol{x}{\ }\hlkeyword{\usebox{\hlnormalsizeboxgreaterthan}}{\ }\hlnumber{10}\hlkeyword{]}{\ }{\ }\hlcomment{\usebox{\hlnormalsizeboxhash}{\ }more{\ }typical{\ }use{\ }of{\ }boolean{\ }in{\ }selection}\mbox{}
\normalfont
\end{flushleft}
\begin{verbatim}
[1] 12 14 16 18 20
\end{verbatim}
\end{kframe}}
\end{knitrout}


\noindent
Here are some more examples.
\Rindex{toupper()}%
\Rindex{tolower()}%

\begin{knitrout}
\definecolor{shadecolor}{rgb}{.97, .97, .97}{\color{fgcolor}\begin{kframe}
\begin{flushleft}
\ttfamily\noindent
\hlsymbol{notes}{\ }\hlassignement{\usebox{\hlnormalsizeboxlessthan}-}{\ }\hlfunctioncall{toupper}\hlkeyword{(}\hlsymbol{letters}\hlkeyword{[}\hlnumber{1}\hlkeyword{:}\hlnumber{7}\hlkeyword{]}\hlkeyword{)}\hspace*{\fill}\\
\hlstd{}\hlsymbol{a}{\ }\hlassignement{\usebox{\hlnormalsizeboxlessthan}-}{\ }\hlnumber{1}\hlkeyword{:}\hlnumber{5}\hspace*{\fill}\\
\hlstd{}\hlsymbol{b}{\ }\hlassignement{\usebox{\hlnormalsizeboxlessthan}-}{\ }\hlfunctioncall{seq}\hlkeyword{(}\hlnumber{10}\hlkeyword{,}{\ }\hlnumber{100}\hlkeyword{,}{\ }\hlargument{by}{\ }\hlargument{=}{\ }\hlnumber{10}\hlkeyword{)}\hspace*{\fill}\\
\hlstd{}\hlfunctioncall{toupper}\hlkeyword{(}\hlsymbol{letters}\hlkeyword{[}\hlnumber{5}\hlkeyword{:}\hlnumber{10}\hlkeyword{]}\hlkeyword{)}\mbox{}
\normalfont
\end{flushleft}
\begin{verbatim}
[1] "E" "F" "G" "H" "I" "J"
\end{verbatim}
\begin{flushleft}
\ttfamily\noindent
\hlfunctioncall{paste}\hlkeyword{(}\hlsymbol{letters}\hlkeyword{[}\hlnumber{1}\hlkeyword{:}\hlnumber{5}\hlkeyword{]}\hlkeyword{,}{\ }\hlnumber{1}\hlkeyword{:}\hlnumber{3}\hlkeyword{,}{\ }\hlargument{sep}{\ }\hlargument{=}{\ }\hlstring{"{}-"{}}\hlkeyword{)}\mbox{}
\normalfont
\end{flushleft}
\begin{verbatim}
[1] "a-1" "b-2" "c-3" "d-1" "e-2"
\end{verbatim}
\begin{flushleft}
\ttfamily\noindent
\hlsymbol{a}{\ }\hlkeyword{+}{\ }\hlsymbol{b}\mbox{}
\normalfont
\end{flushleft}
\begin{verbatim}
 [1]  11  22  33  44  55  61  72  83  94 105
\end{verbatim}
\begin{flushleft}
\ttfamily\noindent
\hlkeyword{(}\hlsymbol{a}{\ }\hlkeyword{+}{\ }\hlsymbol{b}\hlkeyword{)}\hlkeyword{[}\hlsymbol{a}{\ }\hlkeyword{+}{\ }\hlsymbol{b}{\ }\hlkeyword{\usebox{\hlnormalsizeboxgreaterthan}}{\ }\hlnumber{50}\hlkeyword{]}\mbox{}
\normalfont
\end{flushleft}
\begin{verbatim}
[1]  55  61  72  83  94 105
\end{verbatim}
\begin{flushleft}
\ttfamily\noindent
\hlfunctioncall{length}\hlkeyword{(}\hlkeyword{(}\hlsymbol{a}{\ }\hlkeyword{+}{\ }\hlsymbol{b}\hlkeyword{)}\hlkeyword{[}\hlsymbol{a}{\ }\hlkeyword{+}{\ }\hlsymbol{b}{\ }\hlkeyword{\usebox{\hlnormalsizeboxgreaterthan}}{\ }\hlnumber{50}\hlkeyword{]}\hlkeyword{)}\mbox{}
\normalfont
\end{flushleft}
\begin{verbatim}
[1] 6
\end{verbatim}
\begin{flushleft}
\ttfamily\noindent
\hlfunctioncall{table}\hlkeyword{(}\hlsymbol{a}{\ }\hlkeyword{+}{\ }\hlsymbol{b}{\ }\hlkeyword{\usebox{\hlnormalsizeboxgreaterthan}}{\ }\hlnumber{50}\hlkeyword{)}\mbox{}
\normalfont
\end{flushleft}
\begin{verbatim}

FALSE  TRUE 
    4     6 
\end{verbatim}
\end{kframe}}
\end{knitrout}


%\includepdf[pages=35-39,frame]{Paradis-rdebuts_en.pdf}

%\includepdf[pages=36-37,landscape,rotateoversize,turn=false,nup=1x2,frame]{Paradis-rdebuts_en.pdf}

\section{Manipulating Data Frames}
\label{sec:manipulatingData}%

%\subsection{Cross Tabulation with \texttt{xtabs()}}
%\subsection{\texttt{aggregate()} and \texttt{Hmisc::summary()}}
\subsection{Adding new variables to a data frame}
We can add additional variables to an existing data frame by simple assignment.

\begin{knitrout}
\definecolor{shadecolor}{rgb}{.97, .97, .97}{\color{fgcolor}\begin{kframe}
\begin{flushleft}
\ttfamily\noindent
\hlfunctioncall{summary}\hlkeyword{(}\hlsymbol{iris}\hlkeyword{)}\mbox{}
\normalfont
\end{flushleft}
\begin{verbatim}
  Sepal.Length   Sepal.Width    Petal.Length   Petal.Width        Species  
 Min.   :4.30   Min.   :2.00   Min.   :1.00   Min.   :0.1   setosa    :50  
 1st Qu.:5.10   1st Qu.:2.80   1st Qu.:1.60   1st Qu.:0.3   versicolor:50  
 Median :5.80   Median :3.00   Median :4.35   Median :1.3   virginica :50  
 Mean   :5.84   Mean   :3.06   Mean   :3.76   Mean   :1.2                  
 3rd Qu.:6.40   3rd Qu.:3.30   3rd Qu.:5.10   3rd Qu.:1.8                  
 Max.   :7.90   Max.   :4.40   Max.   :6.90   Max.   :2.5                  
\end{verbatim}
\end{kframe}}
\end{knitrout}


\begin{knitrout}
\definecolor{shadecolor}{rgb}{.97, .97, .97}{\color{fgcolor}\begin{kframe}
\begin{flushleft}
\ttfamily\noindent
\hlsymbol{iris}\hlkeyword{\usebox{\hlnormalsizeboxdollar}}\hlsymbol{SLength}{\ }\hlassignement{\usebox{\hlnormalsizeboxlessthan}-}{\ }\hlfunctioncall{cut}\hlkeyword{(}\hlsymbol{iris}\hlkeyword{\usebox{\hlnormalsizeboxdollar}}\hlsymbol{Sepal.Length}\hlkeyword{,}{\ }\hlnumber{4}\hlkeyword{:}\hlnumber{8}\hlkeyword{)}{\ }{\ }\hlcomment{\usebox{\hlnormalsizeboxhash}{\ }cut{\ }places{\ }data{\ }into{\ }bins}\mbox{}
\normalfont
\end{flushleft}
\end{kframe}}
\end{knitrout}


\begin{knitrout}
\definecolor{shadecolor}{rgb}{.97, .97, .97}{\color{fgcolor}\begin{kframe}
\begin{flushleft}
\ttfamily\noindent
\hlfunctioncall{summary}\hlkeyword{(}\hlsymbol{iris}\hlkeyword{)}\mbox{}
\normalfont
\end{flushleft}
\begin{verbatim}
  Sepal.Length   Sepal.Width    Petal.Length   Petal.Width        Species    SLength  
 Min.   :4.30   Min.   :2.00   Min.   :1.00   Min.   :0.1   setosa    :50   (4,5]:32  
 1st Qu.:5.10   1st Qu.:2.80   1st Qu.:1.60   1st Qu.:0.3   versicolor:50   (5,6]:57  
 Median :5.80   Median :3.00   Median :4.35   Median :1.3   virginica :50   (6,7]:49  
 Mean   :5.84   Mean   :3.06   Mean   :3.76   Mean   :1.2                   (7,8]:12  
 3rd Qu.:6.40   3rd Qu.:3.30   3rd Qu.:5.10   3rd Qu.:1.8                             
 Max.   :7.90   Max.   :4.40   Max.   :6.90   Max.   :2.5                             
\end{verbatim}
\end{kframe}}
\end{knitrout}

\Rindex{summary()}

It is an error to add a vector of the wrong length.

The \dfn{CPS} data frame contains data from a Current Population Survey (current in 1985, that is).
Two of the variables in this data frame are \variable{age} and \variable{educ}.  We can estimate
the number of years a worker has been in the workforce if we assume they have been in the workforce
since completing their education and that their age at graduation is 6 more than the number
of years of education obtained.  We can this as a new variable in the data frame simply
by assigning to it:
\begin{knitrout}
\definecolor{shadecolor}{rgb}{.97, .97, .97}{\color{fgcolor}\begin{kframe}
\begin{flushleft}
\ttfamily\noindent
\hlsymbol{CPS}\hlkeyword{\usebox{\hlnormalsizeboxdollar}}\hlsymbol{workforce.years}{\ }\hlassignement{\usebox{\hlnormalsizeboxlessthan}-}{\ }\hlfunctioncall{with}\hlkeyword{(}\hlsymbol{CPS}\hlkeyword{,}{\ }\hlsymbol{age}{\ }\hlkeyword{-}{\ }\hlnumber{6}{\ }\hlkeyword{-}{\ }\hlsymbol{educ}\hlkeyword{)}\hspace*{\fill}\\
\hlstd{}\hlfunctioncall{favstats}\hlkeyword{(}\hlsymbol{CPS}\hlkeyword{\usebox{\hlnormalsizeboxdollar}}\hlsymbol{workforce.years}\hlkeyword{)}\mbox{}
\normalfont
\end{flushleft}
\begin{verbatim}
    0%    25%    50%    75%   100%   mean     sd    var 
 -4.00   8.00  15.00  26.00  55.00  17.81  12.39 153.55 
\end{verbatim}
\end{kframe}}
\end{knitrout}

In fact this is what was done for all but one of the cases to create the \variable{exper} 
variable that is already in the \dfn{CPS} data.
\begin{knitrout}
\definecolor{shadecolor}{rgb}{.97, .97, .97}{\color{fgcolor}\begin{kframe}
\begin{flushleft}
\ttfamily\noindent
\hlfunctioncall{with}\hlkeyword{(}\hlsymbol{CPS}\hlkeyword{,}{\ }\hlfunctioncall{table}\hlkeyword{(}\hlsymbol{exper}{\ }\hlkeyword{-}{\ }\hlsymbol{workforce.years}\hlkeyword{)}\hlkeyword{)}\mbox{}
\normalfont
\end{flushleft}
\begin{verbatim}

  0   4 
533   1 
\end{verbatim}
\end{kframe}}
\end{knitrout}


\subsection{Dropping variables}
Since we already have \variable{educ}, there is no reason to keep our new variable.  Let's drop it.
Notice the clever use of the minus sign.
\begin{knitrout}
\definecolor{shadecolor}{rgb}{.97, .97, .97}{\color{fgcolor}\begin{kframe}
\begin{flushleft}
\ttfamily\noindent
\hlsymbol{CPS1}{\ }\hlassignement{\usebox{\hlnormalsizeboxlessthan}-}{\ }\hlfunctioncall{subset}\hlkeyword{(}\hlsymbol{CPS}\hlkeyword{,}{\ }\hlargument{select}{\ }\hlargument{=}{\ }\hlkeyword{-}\hlsymbol{workforce.years}\hlkeyword{)}\mbox{}
\normalfont
\end{flushleft}
\end{kframe}}
\end{knitrout}

Any number of variables can be dropped or kept in this manner by supplying a vectors
of variables names.
\begin{knitrout}
\definecolor{shadecolor}{rgb}{.97, .97, .97}{\color{fgcolor}\begin{kframe}
\begin{flushleft}
\ttfamily\noindent
\hlsymbol{CPS1}{\ }\hlassignement{\usebox{\hlnormalsizeboxlessthan}-}{\ }\hlfunctioncall{subset}\hlkeyword{(}\hlsymbol{CPS}\hlkeyword{,}{\ }\hlargument{select}{\ }\hlargument{=}{\ }\hlkeyword{-}\hlfunctioncall{c}\hlkeyword{(}\hlsymbol{workforce.years}\hlkeyword{,}{\ }\hlsymbol{exper}\hlkeyword{)}\hlkeyword{)}\mbox{}
\normalfont
\end{flushleft}
\end{kframe}}
\end{knitrout}


If we only want to work with the first few variables, we can discard the rest in a similar way.
Columns can be specified by number as well as name (but this can be dangerous if you are wrong 
about where the columns are):
\begin{knitrout}
\definecolor{shadecolor}{rgb}{.97, .97, .97}{\color{fgcolor}\begin{kframe}
\begin{flushleft}
\ttfamily\noindent
\hlsymbol{CPSsmall}{\ }\hlassignement{\usebox{\hlnormalsizeboxlessthan}-}{\ }\hlfunctioncall{subset}\hlkeyword{(}\hlsymbol{CPS}\hlkeyword{,}{\ }\hlargument{select}{\ }\hlargument{=}{\ }\hlnumber{1}\hlkeyword{:}\hlnumber{4}\hlkeyword{)}\hspace*{\fill}\\
\hlstd{}\hlfunctioncall{head}\hlkeyword{(}\hlsymbol{CPSsmall}\hlkeyword{,}{\ }\hlnumber{2}\hlkeyword{)}\mbox{}
\normalfont
\end{flushleft}
\begin{verbatim}
  wage educ race sex
1  9.0   10    W   M
2  5.5   12    W   M
\end{verbatim}
\end{kframe}}
\end{knitrout}


\subsection{Renaming variables}
Both the column (variable) names and the row names of a data frames can be changed by
simple assignment using \function{names()} or \function{row.names()}.
\begin{knitrout}
\definecolor{shadecolor}{rgb}{.97, .97, .97}{\color{fgcolor}\begin{kframe}
\begin{flushleft}
\ttfamily\noindent
\hlsymbol{ddd}{\ }{\ }\hlcomment{\usebox{\hlnormalsizeboxhash}{\ }small{\ }data{\ }frame{\ }we{\ }defined{\ }earlier}\mbox{}
\normalfont
\end{flushleft}
\begin{verbatim}
  number letter
1      1      a
2      2      b
3      3      c
4      4      d
5      5      e
\end{verbatim}
\begin{flushleft}
\ttfamily\noindent
\hlfunctioncall{row.names}\hlkeyword{(}\hlsymbol{ddd}\hlkeyword{)}{\ }\hlassignement{\usebox{\hlnormalsizeboxlessthan}-}{\ }\hlfunctioncall{c}\hlkeyword{(}\hlstring{"{}Abe"{}}\hlkeyword{,}{\ }\hlstring{"{}Betty"{}}\hlkeyword{,}{\ }\hlstring{"{}Claire"{}}\hlkeyword{,}{\ }\hlstring{"{}Don"{}}\hlkeyword{,}{\ }\hlstring{"{}Ethel"{}}\hlkeyword{)}\hspace*{\fill}\\
\hlstd{}\hlsymbol{ddd}{\ }{\ }\hlcomment{\usebox{\hlnormalsizeboxhash}{\ }row.names{\ }affects{\ }how{\ }a{\ }data.frame{\ }prints}\mbox{}
\normalfont
\end{flushleft}
\begin{verbatim}
       number letter
Abe         1      a
Betty       2      b
Claire      3      c
Don         4      d
Ethel       5      e
\end{verbatim}
\end{kframe}}
\end{knitrout}

More interestingly, it is possible to reset just individual names with the following
syntax.
\begin{knitrout}
\definecolor{shadecolor}{rgb}{.97, .97, .97}{\color{fgcolor}\begin{kframe}
\begin{flushleft}
\ttfamily\noindent
\hlfunctioncall{row.names}\hlkeyword{(}\hlsymbol{ddd}\hlkeyword{)}\hlkeyword{[}\hlnumber{2}\hlkeyword{]}{\ }\hlassignement{\usebox{\hlnormalsizeboxlessthan}-}{\ }\hlstring{"{}Bette"{}}{\ }{\ }\hlcomment{\usebox{\hlnormalsizeboxhash}{\ }misspelled{\ }a{\ }name,{\ }let\usebox{\hlnormalsizeboxsinglequote}s{\ }fix{\ }it}\hspace*{\fill}\\
\hlstd{}\hlfunctioncall{row.names}\hlkeyword{(}\hlsymbol{ddd}\hlkeyword{)}\mbox{}
\normalfont
\end{flushleft}
\begin{verbatim}
[1] "Abe"    "Bette"  "Claire" "Don"    "Ethel" 
\end{verbatim}
\end{kframe}}
\end{knitrout}


The \dfn{faithful} data set (in the \pkg{datasets} package, which is always available)
has very unfortunate names.
\begin{knitrout}
\definecolor{shadecolor}{rgb}{.97, .97, .97}{\color{fgcolor}\begin{kframe}
\begin{flushleft}
\ttfamily\noindent
\hlfunctioncall{names}\hlkeyword{(}\hlsymbol{faithful}\hlkeyword{)}\mbox{}
\normalfont
\end{flushleft}
\begin{verbatim}
[1] "eruptions" "waiting"  
\end{verbatim}
\end{kframe}}
\end{knitrout}

The measurements are the duration of an euption and the time until the subsequent eruption,
so let's give it some better names.
\begin{knitrout}
\definecolor{shadecolor}{rgb}{.97, .97, .97}{\color{fgcolor}\begin{kframe}
\begin{flushleft}
\ttfamily\noindent
\hlfunctioncall{names}\hlkeyword{(}\hlsymbol{faithful}\hlkeyword{)}{\ }\hlassignement{\usebox{\hlnormalsizeboxlessthan}-}{\ }\hlfunctioncall{c}\hlkeyword{(}\hlstring{"{}duration"{}}\hlkeyword{,}{\ }\hlstring{"{}time.til.next"{}}\hlkeyword{)}\hspace*{\fill}\\
\hlstd{}\hlfunctioncall{head}\hlkeyword{(}\hlsymbol{faithful}\hlkeyword{,}{\ }\hlnumber{3}\hlkeyword{)}\mbox{}
\normalfont
\end{flushleft}
\begin{verbatim}
  duration time.til.next
1    3.600            79
2    1.800            54
3    3.333            74
\end{verbatim}
\end{kframe}}
\end{knitrout}

\begin{center}
\begin{knitrout}
\definecolor{shadecolor}{rgb}{.97, .97, .97}{\color{fgcolor}\begin{kframe}
\begin{flushleft}
\ttfamily\noindent
\hlfunctioncall{xyplot}\hlkeyword{(}\hlsymbol{time.til.next}{\ }\hlkeyword{\urltilda{}}{\ }\hlsymbol{duration}\hlkeyword{,}{\ }\hlsymbol{faithful}\hlkeyword{)}\mbox{}
\normalfont
\end{flushleft}


\centering{}\includegraphics{figures/fig-faithful-xy} 

\end{kframe}}
\end{knitrout}

\end{center}
If the variable containing a data frame is modified or used to store a different object,
the original data from the package can be recovered using \function{data()}.
\begin{knitrout}
\definecolor{shadecolor}{rgb}{.97, .97, .97}{\color{fgcolor}\begin{kframe}
\begin{flushleft}
\ttfamily\noindent
\hlfunctioncall{data}\hlkeyword{(}\hlsymbol{faithful}\hlkeyword{)}\hspace*{\fill}\\
\hlstd{}\hlfunctioncall{head}\hlkeyword{(}\hlsymbol{faithful}\hlkeyword{,}{\ }\hlnumber{3}\hlkeyword{)}\mbox{}
\normalfont
\end{flushleft}
\begin{verbatim}
  eruptions waiting
1     3.600      79
2     1.800      54
3     3.333      74
\end{verbatim}
\end{kframe}}
\end{knitrout}


\begin{problem}
Using \dfn{faithful} data frame, make a scatter plot of eruption duration times vs. the time
since the previous eruption.
\end{problem}

If we want to rename a variable, we can do this using \function{names()}.
For example, perhaps we want to rename \variable{educ} (the second column) to \variable{education}.
\begin{knitrout}
\definecolor{shadecolor}{rgb}{.97, .97, .97}{\color{fgcolor}\begin{kframe}
\begin{flushleft}
\ttfamily\noindent
\hlfunctioncall{names}\hlkeyword{(}\hlsymbol{CPS}\hlkeyword{)}\hlkeyword{[}\hlnumber{2}\hlkeyword{]}{\ }\hlassignement{\usebox{\hlnormalsizeboxlessthan}-}{\ }\hlstring{"{}education"{}}\hspace*{\fill}\\
\hlstd{}\hlsymbol{CPS}\hlkeyword{[}\hlnumber{1}\hlkeyword{,}{\ }\hlnumber{1}\hlkeyword{:}\hlnumber{4}\hlkeyword{]}\mbox{}
\normalfont
\end{flushleft}
\begin{verbatim}
  wage education race sex
1    9        10    W   M
\end{verbatim}
\end{kframe}}
\end{knitrout}


If we don't know the column number (or generally to make our code clearer), a few more 
keystrokes produces
\FoodForThought{See Section \ref{sec:Rvectors} for information that will make 
it clearer what is going on here.}
\begin{knitrout}
\definecolor{shadecolor}{rgb}{.97, .97, .97}{\color{fgcolor}\begin{kframe}
\begin{flushleft}
\ttfamily\noindent
\hlfunctioncall{names}\hlkeyword{(}\hlsymbol{CPS}\hlkeyword{)}\hlkeyword{[}\hlfunctioncall{names}\hlkeyword{(}\hlsymbol{CPS}\hlkeyword{)}{\ }=={\ }\hlstring{"{}education"{}}\hlkeyword{]}{\ }\hlassignement{\usebox{\hlnormalsizeboxlessthan}-}{\ }\hlstring{"{}educ"{}}\hspace*{\fill}\\
\hlstd{}\hlsymbol{CPS}\hlkeyword{[}\hlnumber{1}\hlkeyword{,}{\ }\hlnumber{1}\hlkeyword{:}\hlnumber{4}\hlkeyword{]}\mbox{}
\normalfont
\end{flushleft}
\begin{verbatim}
  wage educ race sex
1    9   10    W   M
\end{verbatim}
\end{kframe}}
\end{knitrout}


\subsection{Creating subsets}
\label{sec:subsets}
We can also use \function{subset()} to reduce the size of a data set by selecting 
only certain rows.
\begin{center}
\begin{knitrout}
\definecolor{shadecolor}{rgb}{.97, .97, .97}{\color{fgcolor}\begin{kframe}
\begin{flushleft}
\ttfamily\noindent
\hlfunctioncall{data}\hlkeyword{(}\hlsymbol{faithful}\hlkeyword{)}\hspace*{\fill}\\
\hlstd{}\hlfunctioncall{names}\hlkeyword{(}\hlsymbol{faithful}\hlkeyword{)}{\ }\hlassignement{\usebox{\hlnormalsizeboxlessthan}-}{\ }\hlfunctioncall{c}\hlkeyword{(}\hlstring{"{}duration"{}}\hlkeyword{,}{\ }\hlstring{"{}time.til.next"{}}\hlkeyword{)}\hspace*{\fill}\\
\hlstd{}\hlcomment{\usebox{\hlnormalsizeboxhash}{\ }any{\ }logical{\ }can{\ }be{\ }used{\ }to{\ }create{\ }subsets}\hspace*{\fill}\\
\hlstd{}\hlsymbol{faithfulLong}{\ }\hlassignement{\usebox{\hlnormalsizeboxlessthan}-}{\ }\hlfunctioncall{subset}\hlkeyword{(}\hlsymbol{faithful}\hlkeyword{,}{\ }\hlsymbol{duration}{\ }\hlkeyword{\usebox{\hlnormalsizeboxgreaterthan}}{\ }\hlnumber{3}\hlkeyword{)}\hspace*{\fill}\\
\hlstd{}\hlfunctioncall{xyplot}\hlkeyword{(}\hlsymbol{time.til.next}{\ }\hlkeyword{\urltilda{}}{\ }\hlsymbol{duration}\hlkeyword{,}{\ }\hlsymbol{faithfulLong}\hlkeyword{)}\mbox{}
\normalfont
\end{flushleft}


\centering{}\includegraphics{figures/fig-faithful-long-xy} 

\end{kframe}}
\end{knitrout}

\end{center}

Of course, if all we want to do is produce a graph, there is no reason to create 
a new data frame.  The plot above could also be made with
\begin{knitrout}
\definecolor{shadecolor}{rgb}{.97, .97, .97}{\color{fgcolor}\begin{kframe}
\begin{flushleft}
\ttfamily\noindent
\hlfunctioncall{xyplot}\hlkeyword{(}\hlsymbol{time.til.next}{\ }\hlkeyword{\urltilda{}}{\ }\hlsymbol{duration}\hlkeyword{,}{\ }\hlsymbol{faithful}\hlkeyword{,}{\ }\hlargument{subset}{\ }\hlargument{=}{\ }\hlsymbol{duration}{\ }\hlkeyword{\usebox{\hlnormalsizeboxgreaterthan}}{\ }\hlnumber{3}\hlkeyword{)}\mbox{}
\normalfont
\end{flushleft}
\end{kframe}}
\end{knitrout}



\authNote{rjp: deleted stubs for creating variables and accounting for missing data.
We can add them back in if/when we write them.}%
%\subsection{Creating variables}
%
%% HELP$newsex <- factor(HELP$female, labels=c('M','F'))
%
%\subsubsection{character variables}
%\subsubsection{coercing variables}
%\subsubsection{recoding variables}
%\subsection{Accounting for missing data}

\subsection{Merging datasets}

The \dfn{fusion1} data frame in the \pkg{fastR} package contains 
genotype information for a SNP (single nucleotide polymorphism) in the gene
\emph{TCF7L2}.  
The \dfn{pheno} data frame contains phenotypes
(including type 2 diabetes case/control status) for an intersecting set of individuals.
We can merge these together to explore the association between
genotypes and phenotypes using \verb!merge()!.

%\Rindex{merge()}%
\begin{knitrout}
\definecolor{shadecolor}{rgb}{.97, .97, .97}{\color{fgcolor}\begin{kframe}
\begin{flushleft}
\ttfamily\noindent
\hlfunctioncall{require}\hlkeyword{(}\hlsymbol{fastR}\hlkeyword{)}\hspace*{\fill}\\
\hlstd{}\hlfunctioncall{head}\hlkeyword{(}\hlsymbol{fusion1}\hlkeyword{,}{\ }\hlnumber{3}\hlkeyword{)}\mbox{}
\normalfont
\end{flushleft}
\begin{verbatim}
     id     marker markerID allele1 allele2 genotype Adose Cdose Gdose Tdose
1  9735 RS12255372        1       3       3       GG     0     0     2     0
2 10158 RS12255372        1       3       3       GG     0     0     2     0
3  9380 RS12255372        1       3       4       GT     0     0     1     1
\end{verbatim}
\begin{flushleft}
\ttfamily\noindent
\hlfunctioncall{head}\hlkeyword{(}\hlsymbol{pheno}\hlkeyword{,}{\ }\hlnumber{3}\hlkeyword{)}\mbox{}
\normalfont
\end{flushleft}
\begin{verbatim}
    id     t2d   bmi sex   age smoker chol waist weight height    whr sbp dbp
1 1002    case 32.86   F 70.76 former 4.57 112.0   85.6  161.4 0.9868 135  77
2 1009    case 27.39   F 53.92  never 7.32  93.5   77.4  168.1 0.9397 158  88
3 1012 control 30.47   M 53.86 former 5.02 104.0   94.6  176.2 0.9327 143  89
\end{verbatim}
\end{kframe}}
\end{knitrout}


\begin{knitrout}
\definecolor{shadecolor}{rgb}{.97, .97, .97}{\color{fgcolor}\begin{kframe}
\begin{flushleft}
\ttfamily\noindent
\hlcomment{\usebox{\hlnormalsizeboxhash}{\ }merge{\ }fusion1{\ }and{\ }pheno{\ }keeping{\ }only{\ }id\usebox{\hlnormalsizeboxsinglequote}s{\ }that{\ }are{\ }in{\ }both}\hspace*{\fill}\\
\hlstd{}\hlsymbol{fusion1m}{\ }\hlassignement{\usebox{\hlnormalsizeboxlessthan}-}{\ }\hlfunctioncall{merge}\hlkeyword{(}\hlsymbol{fusion1}\hlkeyword{,}{\ }\hlsymbol{pheno}\hlkeyword{,}{\ }\hlargument{by.x}{\ }\hlargument{=}{\ }\hlstring{"{}id"{}}\hlkeyword{,}{\ }\hlargument{by.y}{\ }\hlargument{=}{\ }\hlstring{"{}id"{}}\hlkeyword{,}{\ }\hlargument{all.x}{\ }\hlargument{=}{\ }\hlnumber{FALSE}\hlkeyword{,}\hspace*{\fill}\\
\hlstd{}{\ }{\ }{\ }{\ }\hlargument{all.y}{\ }\hlargument{=}{\ }\hlnumber{FALSE}\hlkeyword{)}\hspace*{\fill}\\
\hlstd{}\hlfunctioncall{head}\hlkeyword{(}\hlsymbol{fusion1m}\hlkeyword{,}{\ }\hlnumber{3}\hlkeyword{)}\mbox{}
\normalfont
\end{flushleft}
\begin{verbatim}
    id     marker markerID allele1 allele2 genotype Adose Cdose Gdose Tdose     t2d   bmi sex   age
1 1002 RS12255372        1       3       3       GG     0     0     2     0    case 32.86   F 70.76
2 1009 RS12255372        1       3       3       GG     0     0     2     0    case 27.39   F 53.92
3 1012 RS12255372        1       3       3       GG     0     0     2     0 control 30.47   M 53.86
  smoker chol waist weight height    whr sbp dbp
1 former 4.57 112.0   85.6  161.4 0.9868 135  77
2  never 7.32  93.5   77.4  168.1 0.9397 158  88
3 former 5.02 104.0   94.6  176.2 0.9327 143  89
\end{verbatim}
\end{kframe}}
\end{knitrout}

In this case, since the values are the same for each data frame, we could collapse
\option{by.x} and \option{by.y} to \option{by} and collapse
\option{all.x} and \option{all.y} to \option{all}.
The first of these specifies which column(s) to use to identify matching cases.
The second indicates whether cases in one data frame that do not appear in the other 
should be kept (\code{TRUE}) or dropped 
(filling in \code{NA} as needed) or dropped from the merged data frame.

Now we are ready to begin our analysis.
\begin{knitrout}
\definecolor{shadecolor}{rgb}{.97, .97, .97}{\color{fgcolor}\begin{kframe}
\begin{flushleft}
\ttfamily\noindent
\hlfunctioncall{xtabs}\hlkeyword{(}\hlkeyword{\urltilda{}}\hlsymbol{t2d}{\ }\hlkeyword{+}{\ }\hlsymbol{genotype}{\ }\hlkeyword{+}{\ }\hlsymbol{marker}\hlkeyword{,}{\ }\hlsymbol{fusion1m}\hlkeyword{)}\mbox{}
\normalfont
\end{flushleft}
\begin{verbatim}
, , marker = RS12255372

         genotype
t2d        GG  GT  TT
  case    737 375  48
  control 835 309  27

\end{verbatim}
\end{kframe}}
\end{knitrout}


\begin{problem}
The \dfn{fusion2} data set in the \pkg{fastR} package contains genotypes for 
another SNP.  Merge \dfn{fusion1}, \dfn{fusion2}, and \dfn{pheno} into a single data
frame.

Note that \dfn{fusion1} and \dfn{fusion2} have the same columns.
\begin{knitrout}
\definecolor{shadecolor}{rgb}{.97, .97, .97}{\color{fgcolor}\begin{kframe}
\begin{flushleft}
\ttfamily\noindent
\hlfunctioncall{names}\hlkeyword{(}\hlsymbol{fusion1}\hlkeyword{)}\mbox{}
\normalfont
\end{flushleft}
\begin{verbatim}
 [1] "id"       "marker"   "markerID" "allele1"  "allele2"  "genotype" "Adose"    "Cdose"   
 [9] "Gdose"    "Tdose"   
\end{verbatim}
\begin{flushleft}
\ttfamily\noindent
\hlfunctioncall{names}\hlkeyword{(}\hlsymbol{fusion2}\hlkeyword{)}\mbox{}
\normalfont
\end{flushleft}
\begin{verbatim}
 [1] "id"       "marker"   "markerID" "allele1"  "allele2"  "genotype" "Adose"    "Cdose"   
 [9] "Gdose"    "Tdose"   
\end{verbatim}
\end{kframe}}
\end{knitrout}

You may want to use the \option{suffixes} argument to \function{merge()} or rename the variables
after you are done merging to make the resulting data frame easier to navigate.

Tidy up your data frame by dropping any columns that are redundant or that you just don't want to
have in your final data frame.
\end{problem}

\subsection{Slicing and dicing}
\authNote{NH to expand}


&reshape()& provides a flexible way to change the arrangement of data.  
\Rindex{reshape()}%
It was designed for converting between long and wide versions of 
time series data and its arguments are named with that in mind.

A common situation is when we want to convert from a wide form to a 
long form because of a change in perspective about what a unit of 
observation is.  For example, in the \dfn{traffic} data frame, each 
row is a year, and data for multiple states are provided.

\begin{knitrout}
\definecolor{shadecolor}{rgb}{.97, .97, .97}{\color{fgcolor}\begin{kframe}
\begin{flushleft}
\ttfamily\noindent
\hlsymbol{traffic}\mbox{}
\normalfont
\end{flushleft}
\begin{verbatim}
  year cn.deaths   ny   cn   ma   ri
1 1951       265 13.9 13.0 10.2  8.0
2 1952       230 13.8 10.8 10.0  8.5
3 1953       275 14.4 12.8 11.0  8.5
4 1954       240 13.0 10.8 10.5  7.5
5 1955       325 13.5 14.0 11.8 10.0
6 1956       280 13.4 12.1 11.0  8.2
7 1957       273 13.3 11.9 10.2  9.4
8 1958       248 13.0 10.1 11.8  8.6
9 1959       245 12.9 10.0 11.0  9.0
\end{verbatim}
\end{kframe}}
\end{knitrout}

We can reformat this so that each row contains a measurement for a 
single state in one year.

\begin{knitrout}
\definecolor{shadecolor}{rgb}{.97, .97, .97}{\color{fgcolor}\begin{kframe}
\begin{flushleft}
\ttfamily\noindent
\hlsymbol{longTraffic}{\ }\hlassignement{\usebox{\hlnormalsizeboxlessthan}-}{\ }\hlfunctioncall{reshape}\hlkeyword{(}\hlsymbol{traffic}\hlkeyword{[}\hlkeyword{,}{\ }\hlkeyword{-}\hlnumber{2}\hlkeyword{]}\hlkeyword{,}{\ }\hlargument{idvar}{\ }\hlargument{=}{\ }\hlstring{"{}year"{}}\hlkeyword{,}{\ }\hlargument{ids}{\ }\hlargument{=}{\ }\hlfunctioncall{row.names}\hlkeyword{(}\hlsymbol{traffic}\hlkeyword{)}\hlkeyword{,}\hspace*{\fill}\\
\hlstd{}{\ }{\ }{\ }{\ }\hlargument{times}{\ }\hlargument{=}{\ }\hlfunctioncall{names}\hlkeyword{(}\hlsymbol{traffic}\hlkeyword{)}\hlkeyword{[}\hlnumber{3}\hlkeyword{:}\hlnumber{6}\hlkeyword{]}\hlkeyword{,}{\ }\hlargument{timevar}{\ }\hlargument{=}{\ }\hlstring{"{}state"{}}\hlkeyword{,}{\ }\hlargument{varying}{\ }\hlargument{=}{\ }\hlfunctioncall{list}\hlkeyword{(}\hlfunctioncall{names}\hlkeyword{(}\hlsymbol{traffic}\hlkeyword{)}\hlkeyword{[}\hlnumber{3}\hlkeyword{:}\hlnumber{6}\hlkeyword{]}\hlkeyword{)}\hlkeyword{,}\hspace*{\fill}\\
\hlstd{}{\ }{\ }{\ }{\ }\hlargument{v.names}{\ }\hlargument{=}{\ }\hlstring{"{}deathRate"{}}\hlkeyword{,}{\ }\hlargument{direction}{\ }\hlargument{=}{\ }\hlstring{"{}long"{}}\hlkeyword{)}\hspace*{\fill}\\
\hlstd{}\hlfunctioncall{head}\hlkeyword{(}\hlsymbol{longTraffic}\hlkeyword{)}\mbox{}
\normalfont
\end{flushleft}
\begin{verbatim}
        year state deathRate
1951.ny 1951    ny      13.9
1952.ny 1952    ny      13.8
1953.ny 1953    ny      14.4
1954.ny 1954    ny      13.0
1955.ny 1955    ny      13.5
1956.ny 1956    ny      13.4
\end{verbatim}
\end{kframe}}
\end{knitrout}

And now we can reformat the other way, this time having all data for a given state 
form a row in the data frame.
\begin{knitrout}
\definecolor{shadecolor}{rgb}{.97, .97, .97}{\color{fgcolor}\begin{kframe}
\begin{flushleft}
\ttfamily\noindent
\hlsymbol{stateTraffic}{\ }\hlassignement{\usebox{\hlnormalsizeboxlessthan}-}{\ }\hlfunctioncall{reshape}\hlkeyword{(}\hlsymbol{longTraffic}\hlkeyword{,}{\ }\hlargument{direction}{\ }\hlargument{=}{\ }\hlstring{"{}wide"{}}\hlkeyword{,}{\ }\hlargument{v.names}{\ }\hlargument{=}{\ }\hlstring{"{}deathRate"{}}\hlkeyword{,}\hspace*{\fill}\\
\hlstd{}{\ }{\ }{\ }{\ }\hlargument{idvar}{\ }\hlargument{=}{\ }\hlstring{"{}state"{}}\hlkeyword{,}{\ }\hlargument{timevar}{\ }\hlargument{=}{\ }\hlstring{"{}year"{}}\hlkeyword{)}\hspace*{\fill}\\
\hlstd{}\hlsymbol{stateTraffic}\mbox{}
\normalfont
\end{flushleft}
\begin{verbatim}
        state deathRate.1951 deathRate.1952 deathRate.1953 deathRate.1954 deathRate.1955
1951.ny    ny           13.9           13.8           14.4           13.0           13.5
1951.cn    cn           13.0           10.8           12.8           10.8           14.0
1951.ma    ma           10.2           10.0           11.0           10.5           11.8
1951.ri    ri            8.0            8.5            8.5            7.5           10.0
        deathRate.1956 deathRate.1957 deathRate.1958 deathRate.1959
1951.ny           13.4           13.3           13.0           12.9
1951.cn           12.1           11.9           10.1           10.0
1951.ma           11.0           10.2           11.8           11.0
1951.ri            8.2            9.4            8.6            9.0
\end{verbatim}
\end{kframe}}
\end{knitrout}


In simpler cases, &stack()& or &unstack()& may suffice.
\verb!Hmisc! also provides \verb!reShape()! as an alternative 
to \verb!reshape()!.
\Rindex{stack()}%
\Rindex{unstack()}%

%\subsection{Simple Relational Database Operations}
%
%\subsubsection*{Example: Grades/Courses}
%
%Using the grade/courses database, show how to combine data from
%different data frames. [DTK]

%\subsubsection{Example: Merging Genotype and Phenotype Data}
%\label{example:fusion1-glm1}%
%\myindex{FUSION|exampleidx}%
%\myindex{logistic regression}%

\section{Functions in \R} %{An introduction to writing functions}
\label{sec:writingFunctions}
\myindex{functions in {\sf R}}%

%To really customize a &lattice& plot -- and for many other applications in \R\ -- 
%you need to learn how to use the &panel& argument,
%which means you need to learn how to write functions.
Functions in \R\ have several components:
\begin{itemize}
  \item a \rterm{name} (like &histogram&)\footnote{Actually, it is possible to define 
	functions without naming them; and for short functions that are only needed once,
	this can actually be useful.}
  \item
	an ordered list of named \rterm{arguments} that serve as inputs to the function
	\myindex{argument of an R function@argument of an {\sf R} function}%

	These are matched first by name and then by order to the values supplied by
	the call to the function.  This is why we don't always include the argument name
	in our function calls.  On the other hand, the availability of names means that
	we don't have to remember the order in which arguments are listed.

	Arguments often have \rterm{default values} which are used if no value is 
	supplied in the function call.
  \item
	a \rterm{return value}

	This is the output of the function.  It can be assigned to a variable
	using the assignment operator (&=&, &<-&, or &->&).
	\Rindex{->}%
	\Rindex{<-}%
	\Rindex{=}%

  \item
	\rterm{side effects}
	
	A function may do other things (like make a graph or set some preferences) 
	that are not necessarily part of the return value.

\end{itemize}
When you read the help pages for an \R\ function, you will see that they are organized
in sections related to these components.  
The list of arguments appears in the \rterm{Usage} section along 
with any default values.  Details about how the arguments are used appear in the 
\rterm{Arguments} section.  The return value is listed in the \rterm{Value} section.
Any side effects are typically mentioned in the \rterm{Details} section.  

Now let's try writing our own function.  Suppose you frequently wanted to compute
the mean, median, and standard deviation of a distribution.  You could make 
a function to do all three to save some typing.  
Let's name our function  &favstats()&.  
&favstats()& will have one argument, which we are assuming will be a vector of
numeric values.%
\Rindex{favstats()}%
\Rindex{function()}%
\footnote{There are ways to check the \rterm{class} of an argument
to see if it is a data frame, a vector, numeric, etc.  A really robust function
should check to make sure that the values supplied to the arguments are of appropriate
types.}
Here is how we could define it:
\Rindex{favstats()}%

\begin{knitrout}
\definecolor{shadecolor}{rgb}{.97, .97, .97}{\color{fgcolor}\begin{kframe}
\begin{flushleft}
\ttfamily\noindent
\hlsymbol{favstats}{\ }\hlassignement{\usebox{\hlnormalsizeboxlessthan}-}{\ }\hlkeyword{function}\hlkeyword{(}\hlformalargs{x}\hlkeyword{)}{\ }\hlkeyword{\usebox{\hlnormalsizeboxopenbrace}}\hspace*{\fill}\\
\hlstd{}{\ }{\ }{\ }{\ }\hlfunctioncall{mean}\hlkeyword{(}\hlsymbol{x}\hlkeyword{)}\hspace*{\fill}\\
\hlstd{}{\ }{\ }{\ }{\ }\hlfunctioncall{median}\hlkeyword{(}\hlsymbol{x}\hlkeyword{)}\hspace*{\fill}\\
\hlstd{}{\ }{\ }{\ }{\ }\hlfunctioncall{sd}\hlkeyword{(}\hlsymbol{x}\hlkeyword{)}\hspace*{\fill}\\
\hlstd{}\hlkeyword{\usebox{\hlnormalsizeboxclosebrace}}\hspace*{\fill}\\
\hlstd{}\hlfunctioncall{favstats}\hlkeyword{(}\hlkeyword{(}\hlnumber{1}\hlkeyword{:}\hlnumber{20}\hlkeyword{)}\hlkeyword{\usebox{\hlnormalsizeboxhat}}\hlnumber{2}\hlkeyword{)}\mbox{}
\normalfont
\end{flushleft}
\begin{verbatim}
   sd 
127.9 
\end{verbatim}
\end{kframe}}
\end{knitrout}


The first line says that we are defining a function called &favstats()& with one
argument, named &x&.  The lines surrounded by curly braces give the code
to be executed when the function is called.  So our function computes 
the mean, then the median, then the standard deviation of its argument.

But as you see, this doesn't do exactly what we wanted.  So what's going on?  
The value returned by the last line of a function is (by default) returned
by the function to its calling environment, where it is (by default) printed
to the screen so you can see it.  In our case, we computed the mean, median,
and standard deviation, but only the standard deviation is being returned 
by the function and hence displayed.  So this function is just an inefficient
version of &sd()&.  That isn't really what we wanted.

We can use &print()& to print out things along the way if we like.

\begin{knitrout}
\definecolor{shadecolor}{rgb}{.97, .97, .97}{\color{fgcolor}\begin{kframe}
\begin{flushleft}
\ttfamily\noindent
\hlsymbol{favstats}{\ }\hlassignement{\usebox{\hlnormalsizeboxlessthan}-}{\ }\hlkeyword{function}\hlkeyword{(}\hlformalargs{x}\hlkeyword{)}{\ }\hlkeyword{\usebox{\hlnormalsizeboxopenbrace}}\hspace*{\fill}\\
\hlstd{}{\ }{\ }{\ }{\ }\hlfunctioncall{print}\hlkeyword{(}\hlfunctioncall{mean}\hlkeyword{(}\hlsymbol{x}\hlkeyword{)}\hlkeyword{)}\hspace*{\fill}\\
\hlstd{}{\ }{\ }{\ }{\ }\hlfunctioncall{print}\hlkeyword{(}\hlfunctioncall{median}\hlkeyword{(}\hlsymbol{x}\hlkeyword{)}\hlkeyword{)}\hspace*{\fill}\\
\hlstd{}{\ }{\ }{\ }{\ }\hlfunctioncall{print}\hlkeyword{(}\hlfunctioncall{sd}\hlkeyword{(}\hlsymbol{x}\hlkeyword{)}\hlkeyword{)}\hspace*{\fill}\\
\hlstd{}\hlkeyword{\usebox{\hlnormalsizeboxclosebrace}}\hspace*{\fill}\\
\hlstd{}\hlfunctioncall{favstats}\hlkeyword{(}\hlkeyword{(}\hlnumber{1}\hlkeyword{:}\hlnumber{20}\hlkeyword{)}\hlkeyword{\usebox{\hlnormalsizeboxhat}}\hlnumber{2}\hlkeyword{)}\mbox{}
\normalfont
\end{flushleft}
\begin{verbatim}
 mean 
143.5 
median 
 110.5 
   sd 
127.9 
\end{verbatim}
\end{kframe}}
\end{knitrout}


Alternatively, we could use a combination of \verb!cat()! and \verb!paste()!, which
would give us more control over how the output is displayed.
\Rindex{cat()}%
\Rindex{paste()}%

\begin{knitrout}
\definecolor{shadecolor}{rgb}{.97, .97, .97}{\color{fgcolor}\begin{kframe}
\begin{flushleft}
\ttfamily\noindent
\hlsymbol{altfavstats}{\ }\hlassignement{\usebox{\hlnormalsizeboxlessthan}-}{\ }\hlkeyword{function}\hlkeyword{(}\hlformalargs{x}\hlkeyword{)}{\ }\hlkeyword{\usebox{\hlnormalsizeboxopenbrace}}\hspace*{\fill}\\
\hlstd{}{\ }{\ }{\ }{\ }\hlfunctioncall{cat}\hlkeyword{(}\hlfunctioncall{paste}\hlkeyword{(}\hlstring{"{}{\ }{\ }mean:"{}}\hlkeyword{,}{\ }\hlfunctioncall{format}\hlkeyword{(}\hlfunctioncall{mean}\hlkeyword{(}\hlsymbol{x}\hlkeyword{)}\hlkeyword{,}{\ }\hlnumber{4}\hlkeyword{)}\hlkeyword{,}{\ }\hlstring{"{}\usebox{\hlnormalsizeboxbackslash}n"{}}\hlkeyword{)}\hlkeyword{)}\hspace*{\fill}\\
\hlstd{}{\ }{\ }{\ }{\ }\hlfunctioncall{cat}\hlkeyword{(}\hlfunctioncall{paste}\hlkeyword{(}\hlstring{"{}{\ }edian:"{}}\hlkeyword{,}{\ }\hlfunctioncall{format}\hlkeyword{(}\hlfunctioncall{median}\hlkeyword{(}\hlsymbol{x}\hlkeyword{)}\hlkeyword{,}{\ }\hlnumber{4}\hlkeyword{)}\hlkeyword{,}{\ }\hlstring{"{}\usebox{\hlnormalsizeboxbackslash}n"{}}\hlkeyword{)}\hlkeyword{)}\hspace*{\fill}\\
\hlstd{}{\ }{\ }{\ }{\ }\hlfunctioncall{cat}\hlkeyword{(}\hlfunctioncall{paste}\hlkeyword{(}\hlstring{"{}{\ }{\ }{\ }{\ }sd:"{}}\hlkeyword{,}{\ }\hlfunctioncall{format}\hlkeyword{(}\hlfunctioncall{sd}\hlkeyword{(}\hlsymbol{x}\hlkeyword{)}\hlkeyword{,}{\ }\hlnumber{4}\hlkeyword{)}\hlkeyword{,}{\ }\hlstring{"{}\usebox{\hlnormalsizeboxbackslash}n"{}}\hlkeyword{)}\hlkeyword{)}\hspace*{\fill}\\
\hlstd{}\hlkeyword{\usebox{\hlnormalsizeboxclosebrace}}\hspace*{\fill}\\
\hlstd{}\hlfunctioncall{altfavstats}\hlkeyword{(}\hlkeyword{(}\hlnumber{1}\hlkeyword{:}\hlnumber{20}\hlkeyword{)}\hlkeyword{\usebox{\hlnormalsizeboxhat}}\hlnumber{2}\hlkeyword{)}\mbox{}
\normalfont
\end{flushleft}
\begin{verbatim}
  mean: 143.5 
 edian: 110.5 
    sd: 127.9 
\end{verbatim}
\end{kframe}}
\end{knitrout}

Either of these methods will allow us to see all three values, 
but if we try to store them \dots
\authNote{Talk about \function{paste()} some more somewhere?}%

\begin{knitrout}
\definecolor{shadecolor}{rgb}{.97, .97, .97}{\color{fgcolor}\begin{kframe}
\begin{flushleft}
\ttfamily\noindent
\hlsymbol{temp}{\ }\hlassignement{\usebox{\hlnormalsizeboxlessthan}-}{\ }\hlfunctioncall{favstats}\hlkeyword{(}\hlkeyword{(}\hlnumber{1}\hlkeyword{:}\hlnumber{20}\hlkeyword{)}\hlkeyword{\usebox{\hlnormalsizeboxhat}}\hlnumber{2}\hlkeyword{)}\mbox{}
\normalfont
\end{flushleft}
\begin{verbatim}
 mean 
143.5 
median 
 110.5 
   sd 
127.9 
\end{verbatim}
\begin{flushleft}
\ttfamily\noindent
\hlsymbol{temp}\mbox{}
\normalfont
\end{flushleft}
\begin{verbatim}
   sd 
127.9 
\end{verbatim}
\end{kframe}}
\end{knitrout}

A function in \R\ can only have one return value, and by default it is the 
value of the last line in the function.  
In the preceding example we only get the standard deviation since 
that is the value we calculated last.

We would really like the function to return all three summary statistics.  
Our solution will be to
store all three in a vector and return the vector.%
\footnote{If the values had not all been of the same mode, we 
could have used a list instead.}

\begin{knitrout}
\definecolor{shadecolor}{rgb}{.97, .97, .97}{\color{fgcolor}\begin{kframe}
\begin{flushleft}
\ttfamily\noindent
\hlsymbol{favstats}{\ }\hlassignement{\usebox{\hlnormalsizeboxlessthan}-}{\ }\hlkeyword{function}\hlkeyword{(}\hlformalargs{x}\hlkeyword{)}{\ }\hlkeyword{\usebox{\hlnormalsizeboxopenbrace}}\hspace*{\fill}\\
\hlstd{}{\ }{\ }{\ }{\ }\hlfunctioncall{c}\hlkeyword{(}\hlfunctioncall{mean}\hlkeyword{(}\hlsymbol{x}\hlkeyword{)}\hlkeyword{,}{\ }\hlfunctioncall{median}\hlkeyword{(}\hlsymbol{x}\hlkeyword{)}\hlkeyword{,}{\ }\hlfunctioncall{sd}\hlkeyword{(}\hlsymbol{x}\hlkeyword{)}\hlkeyword{)}\hspace*{\fill}\\
\hlstd{}\hlkeyword{\usebox{\hlnormalsizeboxclosebrace}}\hspace*{\fill}\\
\hlstd{}\hlfunctioncall{favstats}\hlkeyword{(}\hlkeyword{(}\hlnumber{1}\hlkeyword{:}\hlnumber{20}\hlkeyword{)}\hlkeyword{\usebox{\hlnormalsizeboxhat}}\hlnumber{2}\hlkeyword{)}\mbox{}
\normalfont
\end{flushleft}
\begin{verbatim}
  mean median     sd 
 143.5  110.5  127.9 
\end{verbatim}
\end{kframe}}
\end{knitrout}

Now the only problem is that we have to remember which number is which.
We can fix this by giving names to the slots in our vector.
While we're at it, let's add a few more favorites to the list.
We'll also add an explicit &return()&.
\Rindex{return()}%

\begin{knitrout}
\definecolor{shadecolor}{rgb}{.97, .97, .97}{\color{fgcolor}\begin{kframe}
\begin{flushleft}
\ttfamily\noindent
\hlsymbol{favstats}{\ }\hlassignement{\usebox{\hlnormalsizeboxlessthan}-}{\ }\hlkeyword{function}\hlkeyword{(}\hlformalargs{x}\hlkeyword{)}{\ }\hlkeyword{\usebox{\hlnormalsizeboxopenbrace}}\hspace*{\fill}\\
\hlstd{}{\ }{\ }{\ }{\ }\hlsymbol{result}{\ }\hlassignement{\usebox{\hlnormalsizeboxlessthan}-}{\ }\hlfunctioncall{c}\hlkeyword{(}\hlfunctioncall{min}\hlkeyword{(}\hlsymbol{x}\hlkeyword{)}\hlkeyword{,}{\ }\hlfunctioncall{max}\hlkeyword{(}\hlsymbol{x}\hlkeyword{)}\hlkeyword{,}{\ }\hlfunctioncall{mean}\hlkeyword{(}\hlsymbol{x}\hlkeyword{)}\hlkeyword{,}{\ }\hlfunctioncall{median}\hlkeyword{(}\hlsymbol{x}\hlkeyword{)}\hlkeyword{,}{\ }\hlfunctioncall{sd}\hlkeyword{(}\hlsymbol{x}\hlkeyword{)}\hlkeyword{)}\hspace*{\fill}\\
\hlstd{}{\ }{\ }{\ }{\ }\hlfunctioncall{names}\hlkeyword{(}\hlsymbol{result}\hlkeyword{)}{\ }\hlassignement{\usebox{\hlnormalsizeboxlessthan}-}{\ }\hlfunctioncall{c}\hlkeyword{(}\hlstring{"{}min"{}}\hlkeyword{,}{\ }\hlstring{"{}max"{}}\hlkeyword{,}{\ }\hlstring{"{}mean"{}}\hlkeyword{,}{\ }\hlstring{"{}median"{}}\hlkeyword{,}{\ }\hlstring{"{}sd"{}}\hlkeyword{)}\hspace*{\fill}\\
\hlstd{}{\ }{\ }{\ }{\ }\hlfunctioncall{return}\hlkeyword{(}\hlsymbol{result}\hlkeyword{)}\hspace*{\fill}\\
\hlstd{}\hlkeyword{\usebox{\hlnormalsizeboxclosebrace}}\hspace*{\fill}\\
\hlstd{}\hlfunctioncall{favstats}\hlkeyword{(}\hlkeyword{(}\hlnumber{1}\hlkeyword{:}\hlnumber{20}\hlkeyword{)}\hlkeyword{\usebox{\hlnormalsizeboxhat}}\hlnumber{2}\hlkeyword{)}\mbox{}
\normalfont
\end{flushleft}
\begin{verbatim}
   min    max   mean median     sd 
   1.0  400.0  143.5  110.5  127.9 
\end{verbatim}
\begin{flushleft}
\ttfamily\noindent
\hlfunctioncall{summary}\hlkeyword{(}\hlsymbol{Sepal.Length}{\ }\hlkeyword{\urltilda{}}{\ }\hlsymbol{Species}\hlkeyword{,}{\ }\hlargument{data}{\ }\hlargument{=}{\ }\hlsymbol{iris}\hlkeyword{,}{\ }\hlargument{fun}{\ }\hlargument{=}{\ }\hlsymbol{favstats}\hlkeyword{)}\mbox{}
\normalfont
\end{flushleft}
\begin{verbatim}
Sepal.Length    N=150

+-------+----------+---+---+---+-----+------+------+
|       |          |N  |min|max|mean |median|sd    |
+-------+----------+---+---+---+-----+------+------+
|Species|setosa    | 50|4.3|5.8|5.006|5.0   |0.3525|
|       |versicolor| 50|4.9|7.0|5.936|5.9   |0.5162|
|       |virginica | 50|4.9|7.9|6.588|6.5   |0.6359|
+-------+----------+---+---+---+-----+------+------+
|Overall|          |150|4.3|7.9|5.843|5.8   |0.8281|
+-------+----------+---+---+---+-----+------+------+
\end{verbatim}
\begin{flushleft}
\ttfamily\noindent
\hlfunctioncall{aggregate}\hlkeyword{(}\hlsymbol{Sepal.Length}{\ }\hlkeyword{\urltilda{}}{\ }\hlsymbol{Species}\hlkeyword{,}{\ }\hlargument{data}{\ }\hlargument{=}{\ }\hlsymbol{iris}\hlkeyword{,}{\ }\hlargument{FUN}{\ }\hlargument{=}{\ }\hlsymbol{favstats}\hlkeyword{)}\mbox{}
\normalfont
\end{flushleft}
\begin{verbatim}
     Species Sepal.Length.min Sepal.Length.max Sepal.Length.mean Sepal.Length.median
1     setosa           4.3000           5.8000            5.0060              5.0000
2 versicolor           4.9000           7.0000            5.9360              5.9000
3  virginica           4.9000           7.9000            6.5880              6.5000
  Sepal.Length.sd
1          0.3525
2          0.5162
3          0.6359
\end{verbatim}
\end{kframe}}
\end{knitrout}


Notice how nicely this works with \function{aggregate()} and with the \function{summary()} 
function from the \pkg{Hmisc} package.
You can, of course, define your own favorite function to use with \function{summary()}.
\Rindex{favstats()}%
The \function{favstats()} function in the \pkg{mosaic} package includes the 
quartiles, mean, standard, deviation, sample size and number of missing observations.
\FoodForThought{Notice the use of \code{::} to select 
the \function{favstats} function from the \pkg{mosaic}
package rather than the one we just defined.}%
\begin{knitrout}
\definecolor{shadecolor}{rgb}{.97, .97, .97}{\color{fgcolor}\begin{kframe}
\begin{flushleft}
\ttfamily\noindent
\hlpackage{mosaic}\hlkeyword{::}\hlfunctioncall{favstats}\hlkeyword{(}\hlfunctioncall{rnorm}\hlkeyword{(}\hlnumber{100}\hlkeyword{)}\hlkeyword{)}\mbox{}
\normalfont
\end{flushleft}
\begin{verbatim}
    min      Q1  median     Q3   max    mean     sd   n missing
 -2.607 -0.6012 -0.0893 0.5085 2.923 -0.0897 0.9326 100       0
\end{verbatim}
\end{kframe}}
\end{knitrout}


\authNote{Add a problem to create a function to \dots}


\section*{Exercises}

\shipoutProblems

\Rindex{summary()}%
\Rindex{Hmisc}%



\UndefineShortVerb{\&}
