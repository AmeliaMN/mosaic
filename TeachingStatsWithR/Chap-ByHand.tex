\chapter{Computing by Hand}

\label{chap:byhand}








R is a valuable tool, but for an instructor it is just one tool among
many for developing students' skills and understanding.   In this
section, we illustrate some teaching strategies we have found effective
that involve students doing the calculations themselves.  In some
cases, these are actual paper-and-pencil calculations.  In others, we
take a calculation or method apart, using the computer as if it were a set
of hand tools and having the student build the operation.  Still other
examples are kinesthetic in nature: giving students a physical
intuition about the meaning of an operation.  And in others, we use
the students in class collectively to illustrate variation and its meaning.

Since this is a book about using the computer in teaching, it may seem
odd to include ``by hand'' methods.  One reason we do it is to
emphasize that the broad goal is to teach statistics and statistical skills
and concepts.  But another reason has directly to do with building
computing skills and the attitudes that support the skilled us of
computing.  Among these are:
\begin{enumerate}
  \item The computer is not magical.  It's doing the same things you
    can do yourself, but with greater speed and precision and with
    much, much greater toleration of tedious repetition.
  \item That complicated operations are constructed by putting
    together simpler operations.  
  \item That you should be able to open the ``black box'' and confirm
    that the results the computer is giving you make sense.
\end{enumerate}





%\section{Some Simple Model Calculations}

%Group means.  

%\section{Mites and Wilt Disease}

%\authNote{NH to flesh out}


%\section{A demonstration of shuffling and how it implements the null hypothesis}

%\section{Working through an ANOVA calculation by hand}

%\section{Coverage}

%\section{The perils of multiple comparisons}

%\section{Taking Your Class for a Random Walk}
