\chapter{Getting Started: The First Week With R }









\section{Getting Students Familiar with R}

\subsection{Strategies}
\begin{enumerate}
\item
Start right away.

Do something with \R\ on day 1.  Do something else on day 2.  Have students do something by the end of week 1 at the latest.


\item Illustrate frequently.

Have \R\ running every class period and use it as needed throughout the course so students
can see what \R\ does.  Preview topics by showing before asking students to do things.

\item
Teach \R\ as a programming language. (But don't overdo it.)

There is a bit of syntax to learn -- so teach it explicitly.
\begin{itemize}
\item
Capitalization (and spelling) matter
\item
Explain carefully (and repeatedly) the syntax of functions.
\item
Every object in \R\ has a type (class).  Ask frequently: \emph{What type of thing is this?}
\item
Get students to think about what arguments are needed for functions by asking
\emph{What does this function need to know to do its job?}
\end{itemize}
Give more language details in higher level courses
\begin{itemize}
\item
More about \R\ classes
\item
User-defined functions
\item
Control structures such as loops and conditionals.
\end{itemize}

\item ``Less volume, more creativity."  [Mike McCarthy, head coach, Green Bay Packers]

Use a few methods frequently and students will learn how to use them well, flexibly, 
even creatively. 

Focus on a small number of data types: numerical vectors, character
strings, factors, and data frames.  

Not everything needs to be introduced from first principles.  For
instance, categorical variables are easily enough understood by
putting together simple concepts about character strings and vectors.

\item
Find a way to have computers available for tests.

It makes the test match the rest of the course and is a great motivator for students
to learn R.  It also changes what you can ask for and about on tests.

Randy began doing this when his students asked him if there was a way to use computers
during the test ``\emph{since that's how we do all the homework}."  He has students 
bring laptops to class.  Nick has both in-class (without computer) and out-of-class
(take home) components to his assessment.  


\item
Rethink your course.

If you have taught computer-free or computer-light courses in the past, you may need
to rethink some things.  With ubiquitous computing, some things disappear from your 
course:
\begin{itemize}
\item
Reading statistical tables.  

Does anyone still consult a table for values of $\sin$,
or $\log$?  
All of us have sworn off the use of tabulations of critical values of
 distributions (since none of us use them in our professional work, why would we
teach this to students?)

\item
``Computational formulas".

Replace them with computation.  Teach only the most intuitive formulas. Focus on
how they lead to intuition and understanding, \emph{not} computation.

\item
(Most) hand calculations.

\end{itemize}
At the same time, other things become possible that were not before:
\begin{itemize}
\item
Large data sets
\item
Beautiful plots
\item
Simulation/randomization/resampling based methods
\item
Quick computations
\item
Increased focus on concepts rather than calculations
\end{itemize}
Get your students to think that using the computer is just part of how statistics is
done, rather than an add-on.   

\item Keep the message as simple as possible and keep the commands
  accordingly simple.  Particularly when doing graphics, beware of distracting
  students with the sometimes intricate details of beautifying for
  publication.  If the default behavior is good enough, go with it.

\item
Anticipate computationally challenged students, but don't give in.

Some students pick up \R\ very easily.  In every course there will be a few students who
struggle.  Be prepared to help them, but don't  spend time listening to their complaints.
Focus on diagnosing what they don't know and how to help them ``get it''.

Tell students to copy and paste \R\ code and error messages into email when they
have trouble.  When you reply, explain how the error message helped you diagnose their
problem and help them generalize your solution to other situations.
\TeachingTip{Tell your students to copy and paste error messages
into email rather than describe them vaguely.  It's a big time saver for
both of you.}%
\end{enumerate}

\subsection{Tactics}

\begin{enumerate}
\item
Introduce Graphics Early.


Do graphics very early, so that students see that they can get
impressive output from simple commands.  Try to break away from their
prior expectation that there is a ``steep learning curve."

Accept the defaults -- don't worry about the niceties (good labels,
nice breaks on histograms, colors) too early.  Let them become
comfortable with the basic graphics commands and then play (make sure
it feels like play!) with fancying things up.  

Keep in mind that just because the graphs are easy to make on the computer doesn't 
mean your students understand how to read the graphs.  
Use examples that will
help students develop good habits for visualizing data.  Remember:

\begin{center}
\emph{
Students must learn to see before they can see to learn.}  -- R. Pruim
\end{center}

\item Introduce Sampling and Randomization Early.

Since sampling drives much of the logic of statistics, introduce the idea of a random sample very
early, and have students construct their own random samples.  The
phenomenon of a sampling distribution can be introduced in an
intuitive way, setting it up as a topic for later discussion and analysis.


\end{enumerate}

\section{ Examples for Early in the Course }
The remainder of this chapter has some of our favorite activities for early 
in the course.  
\authNote{need to edit}


\subsection{Coins and Cups: The Lady Tasting Tea}
\label{sec:lady-tasting-tea}

\begin{quote}
\emph{This section is a slightly modified version of a handout R. Pruim
has given Intro Stats students on Day 1 \underline{after} going through
the activity as a class discussion.  
}
\end{quote}


There is a famous story about a lady who claimed that
tea with milk tasted different depending on whether the milk was 
added to the tea or the tea added to the milk.
The story is famous because of the setting in which she made this claim.  
She was attending a party in Cambridge, England, in the $1920$s.
Also in attendance were a number of university dons and their wives.
The scientists in attendance scoffed at the woman and her claim.
What, after all, could be the difference?

\myindex{Fisher, R. A.}%
All the scientists but one, that is.  Rather than simply dismiss the 
woman's claim, he proposed that they decide how one 
should \emph{test} the claim.  The tenor of the conversation changed at 
this suggestion, and the scientists began to discuss how the claim should be 
tested.  Within a few minutes cups of tea with milk had been prepared and 
presented to the woman for tasting.

\iffalse
Let's take this simple example as a prototype for a statistical study.
What steps are involved?  
\begin{enumerate}
  \item Determine the question of interest.

	Just what is it we want to know?  It may take some effort to 
	make a vague idea precise.  The precise questions may not exactly
	correspond to our vague questions, and the very exercise of 
	stating the question precisely may modify our question.  Sometimes
	we cannot come up with any way to answer the question we really want
	to answer, so we have to live with some other question that is 
	not exactly what we wanted but is something we can study and will
	(we hope) give us some information about our original question.

	In our example this question seems fairly easy to state:
	Can the lady tell the difference between the two tea preparations?
	But we need to refine this question.  For example, are we 
	asking if she \emph{always} correctly identifies cups of tea
	or merely if she does better than we could do ourselves (by 
	guessing)?  

  \item 
	Determine the \term{population}. 
	\myindex{population}%

	Just who or what do we want to know about?  Are we only interested in
	this one woman or women in general or only women who claim to
	be able to distinguish tea preparations?

  \item
	Select \term{measurements}.

	We are going to need some data.  
	We get our data by making some measurements.
	These might be physical measurements with some device (like a ruler
	or a scale).
	But there are other sorts of measurements too, 
	like the answer to a question on a form.
	Sometimes it is tricky to figure out just what to measure.
	(How do we measure happiness or intelligence, for example?)
	Just how we do our measuring will have important consequences 
	for the subsequent statistical analysis.
	The recorded values of these measurements are called
	\term{variables} (because the values vary from one individual to another).

	In our example, a measurement may consist of recording for a given
	cup of tea whether the woman's claim is correct or incorrect.
	%it is milk-into-tea or tea-into-milk.

  \item
	Determine the \term{sample}.
	\myindex{sample}%

	Usually we cannot measure every individual in our population; we have 
	to select some to measure.  
	But how many and which ones?  
	These are important questions that must be answered.
	Generally speaking, bigger is better, but it is also more expensive.
	Moreover, no size is large enough if the sample is selected inappropriately.

	Suppose we gave the lady one cup of tea.  If she correctly identifies
	the mixing procedure, will we be convinced of her claim?  She might just
	be guessing; so we should probably have her taste more than one 
	cup.  Will we be convinced if she correctly identifies $5$ cups? $10$ cups?
	$50$ cups?

	What if she makes a mistake?  If we present her with $10$ cups and she
	correctly identifies $9$ of the $10$, what will we conclude?  
	\authNote{added A success rate of -- 2010-10-23}%
	A success rate of $90$\% is, it seems,
	much better than just guessing, and anyone can make a mistake now and then.
	But what if she correctly identifies $8$ out of $10$? $80$ out of $100$?
	
	And how should we prepare the cups?  Should we make $5$ each way?  
	\authNote{Left it as "and how" -- 2010-10-23}%
	Does it matter if we tell the woman that there are $5$ prepared 
	each way?
	Should we flip a coin to decide even if that means we might end 
	up with $3$ prepared one way and $7$ the other way?  
	Do any of these differences matter?

  \item
	Make and record the measurements.

	Once we have the design figured out, we have to do the legwork of 
	data collection.  This can be a time-consuming and tedious process.
	In the case of the lady tasting tea, the scientists decided to 
	present her with ten cups of tea which were quickly prepared.
	A study of public opinion may require many thousands of phone calls or 
	personal interviews.
	In a laboratory setting, each measurement might be the result 
	of a carefully performed laboratory experiment.

  \item Organize the data.

	Once the data have been collected, it is often necessary or useful
	to organize them.  Data are typically stored in spreadsheets or 
	in other formats that are convenient for processing with 
	statistical packages.  Very large data sets are often stored in 
	databases.  
	
	Part of the organization of the data may involve producing graphical and
	numerical summaries of the data.  These summaries may give us initial
	insights into our questions or help us detect errors that may have occurred
	to this point.

  \item Draw conclusions from data.

	Once the data have been collected, organized, and analyzed, we need
	to reach a conclusion.  
	Do we believe the woman's claim?  
	Or do we think she is merely guessing?  How sure are we that this
	conclusion is correct?

%	In Parts~\ref{part:inf1}--\ref{part:inf2} 
	Eventually we will
	learn a number of important and frequently used methods for 
	drawing inferences from data.  More importantly, we will learn
	the basic framework used for such procedures so that it should 
	become easier and easier to learn new procedures as we become 
	familiar with the framework.
	
	%How strongly do we believe it?  

  \item Produce a report.

		Typically the results of a statistical study are reported in 
		some manner.  This may be as a refereed article in an academic 
		journal, as an internal report to a company, or as a solution
		to a problem on a homework assignment.  These reports may themselves
		be further distilled into press releases, newspaper articles,
		advertisements, and the like.  The mark of a good report
		is that it provides the essential information about each 
		of the steps of the study.

		As we go along, we will learn some of the standard terminology and
		procedures that you are likely to see in basic statistical reports and 
		will gain a framework for learning more.  
\end{enumerate}

\fi

At this point, you may be wondering who the innovative scientist was and 
what the results of the experiment were.
\myindex{Fisher, R. A.}%
The scientist was R. A. Fisher, who first described this situation
as a pedagogical example in his 1925 book on 
statistical methodology \cite{Fisher:1925:Methods}.
Fisher developed statistical methods that are among the most
important and widely used methods to this day, and most of his 
applications were biological.
\nocite{Fisher:1970:Methods}%

You might also be curious about how the experiment came out.
How many cups of tea were prepared?  How many did the woman 
correctly identify?  What was the conclusion?

Fisher never says.  In his book he is interested in the method, not the 
particular results.  But let's suppose we decide to test the lady with
ten cups of tea.  
We'll flip a coin to decide which way to prepare the cups.  
If we flip a head, we will pour the milk in first; if tails, we 
put the tea in first.
Then we present the ten cups to the lady and have her state which ones she
thinks were prepared each way.  

It is easy to give her a score (9 out of 10, or 7 out of 10, or whatever
it happens to be).  It is trickier to figure out what to do with her score.
Even if she is just guessing and has no idea, she could get lucky and 
get quite a few correct -- maybe even all 10.  But how likely is that?

Let's try an experiment.  I'll flip 10 coins.  You guess which are heads and
which are tails, and we'll see how you do.  

$\vdots$

Comparing with your classmates, we will undoubtedly see that some 
of you did better and others worse.

Now let's suppose the lady gets 9 out of 10 correct.  That's not perfect,
but it is better than we would expect for someone who was just guessing.
On the other hand, it is not impossible to get 9 out of 10 just by guessing.
So here is Fisher's great idea:  Let's figure out how hard it is to get
9 out of 10 by guessing.  If it's not so hard to do, then perhaps that's 
just what happened, so we won't be too impressed with the lady's tea tasting
ability.  On the other hand, if it is really unusual to get 9 out of 10 
correct by guessing, then we will have some evidence that she must 
be able to tell something.

But how do we figure out how unusual it is to get 9 out of 10 just by 
guessing?  We'll learn another method later, but for now, let's just 
flip a bunch of coins and keep track.  If the lady is just guessing, she 
might as well be flipping a coin.

So here's the plan.  We'll flip 10 coins.  We'll call the heads correct 
guesses and the tails incorrect guesses.  Then we'll flip 10 more coins,
and 10 more, and 10 more, and \dots.  That would get pretty tedious.
Fortunately, computers are good at tedious things, so we'll let the computer 
do the flipping for us.

The \verb!rflip()! function can flip one coin

\begin{knitrout}
\definecolor{shadecolor}{rgb}{.97, .97, .97}{\color{fgcolor}\begin{kframe}
\begin{flushleft}
\ttfamily\noindent
\hlfunctioncall{require}\hlkeyword{(}\hlsymbol{mosaic}\hlkeyword{)}\hspace*{\fill}\\
\hlstd{}\hlfunctioncall{rflip}\hlkeyword{(}\hlkeyword{)}\mbox{}
\normalfont
\end{flushleft}
\begin{verbatim}

Flipping 1 coins [ Prob(Heads) = 0.5 ] ...

T

Result: 0 heads.

\end{verbatim}
\end{kframe}}
\end{knitrout}

or a number of coins
\begin{knitrout}
\definecolor{shadecolor}{rgb}{.97, .97, .97}{\color{fgcolor}\begin{kframe}
\begin{flushleft}
\ttfamily\noindent
\hlfunctioncall{rflip}\hlkeyword{(}\hlnumber{10}\hlkeyword{)}\mbox{}
\normalfont
\end{flushleft}
\begin{verbatim}

Flipping 10 coins [ Prob(Heads) = 0.5 ] ...

T H T H T H H T H T

Result: 5 heads.

\end{verbatim}
\end{kframe}}
\end{knitrout}

%and show us the results.

Typing \verb!rflip(10)! a bunch of times is almost as tedious as 
flipping all those coins.   But it is not too hard to 
tell \R\ to \verb!do()! this a bunch of times.
\begin{knitrout}
\definecolor{shadecolor}{rgb}{.97, .97, .97}{\color{fgcolor}\begin{kframe}
\begin{flushleft}
\ttfamily\noindent
\hlfunctioncall{do}\hlkeyword{(}\hlnumber{3}\hlkeyword{)}{\ }\hlkeyword{*}{\ }\hlfunctioncall{rflip}\hlkeyword{(}\hlnumber{10}\hlkeyword{)}\mbox{}
\normalfont
\end{flushleft}
\begin{verbatim}
   n heads tails
1 10     6     4
2 10     6     4
3 10     4     6
\end{verbatim}
\end{kframe}}
\end{knitrout}


Let's get \R\ to \verb!do()! it for us 10,000 times and make 
a table of the results.




\begin{knitrout}
\definecolor{shadecolor}{rgb}{.97, .97, .97}{\color{fgcolor}\begin{kframe}
\begin{flushleft}
\ttfamily\noindent
\hlsymbol{random.ladies}{\ }\hlassignement{\usebox{\hlnormalsizeboxlessthan}-}{\ }\hlfunctioncall{do}\hlkeyword{(}\hlnumber{10000}\hlkeyword{)}{\ }\hlkeyword{*}{\ }\hlfunctioncall{rflip}\hlkeyword{(}\hlnumber{10}\hlkeyword{)}\mbox{}
\normalfont
\end{flushleft}
\end{kframe}}
\end{knitrout}


\begin{knitrout}
\definecolor{shadecolor}{rgb}{.97, .97, .97}{\color{fgcolor}\begin{kframe}
\begin{flushleft}
\ttfamily\noindent
\hlfunctioncall{table}\hlkeyword{(}\hlsymbol{random.ladies}\hlkeyword{\usebox{\hlnormalsizeboxdollar}}\hlsymbol{heads}\hlkeyword{)}\mbox{}
\normalfont
\end{flushleft}
\begin{verbatim}

   0    1    2    3    4    5    6    7    8    9   10 
   5  102  467 1203 2048 2470 2035 1140  415  108    7 
\end{verbatim}
\begin{flushleft}
\ttfamily\noindent
\hlfunctioncall{perctable}\hlkeyword{(}\hlsymbol{random.ladies}\hlkeyword{\usebox{\hlnormalsizeboxdollar}}\hlsymbol{heads}\hlkeyword{)}{\ }{\ }\hlcomment{\usebox{\hlnormalsizeboxhash}{\ }display{\ }table{\ }using{\ }percentages}\mbox{}
\normalfont
\end{flushleft}
\begin{verbatim}

    0     1     2     3     4     5     6     7     8     9    10 
 0.05  1.02  4.67 12.03 20.48 24.70 20.35 11.40  4.15  1.08  0.07 
\end{verbatim}
\end{kframe}}
\end{knitrout}


We can display this table graphically using a plot called a \term{histogram}.
%\vspace{-8mm}
\begin{center}
\begin{knitrout}
\definecolor{shadecolor}{rgb}{.97, .97, .97}{\color{fgcolor}\begin{kframe}
\begin{flushleft}
\ttfamily\noindent
\hlfunctioncall{histogram}\hlkeyword{(}\hlkeyword{\urltilda{}}\hlsymbol{heads}\hlkeyword{,}{\ }\hlsymbol{random.ladies}\hlkeyword{,}{\ }\hlargument{breaks}{\ }\hlargument{=}{\ }\hlkeyword{-}\hlnumber{0.5}{\ }\hlkeyword{+}{\ }\hlkeyword{(}\hlnumber{0}\hlkeyword{:}\hlnumber{11}\hlkeyword{)}\hlkeyword{)}\mbox{}
\normalfont
\end{flushleft}


\centering{}\includegraphics{figures/fig-lady-hist} 

\end{kframe}}
\end{knitrout}

\end{center}
We have control of the histogram display by 
specifying the breaks from -.5 to 10.5:
\begin{knitrout}
\definecolor{shadecolor}{rgb}{.97, .97, .97}{\color{fgcolor}\begin{kframe}
\begin{flushleft}
\ttfamily\noindent
\hlkeyword{-}\hlnumber{0.5}{\ }\hlkeyword{+}{\ }\hlkeyword{(}\hlnumber{0}\hlkeyword{:}\hlnumber{11}\hlkeyword{)}\mbox{}
\normalfont
\end{flushleft}
\begin{verbatim}
 [1] -0.5  0.5  1.5  2.5  3.5  4.5  5.5  6.5  7.5  8.5  9.5 10.5
\end{verbatim}
\end{kframe}}
\end{knitrout}


You might be surprised to see that the number of correct guesses
is exactly 5 (half of the 10 tries) only 
25\%
of the time.  But most of the results are quite close to 5 correct.
67\% of the results are 
4, 5, or 6, for example.
And 90\% of the results 
are  between 3 and 7 (inclusive).
But getting 8 correct is a bit unusual, and getting 9 or 10 correct is even 
more unusual.  

So what do we conclude?  It is possible that the lady could get 9 or 10 correct
just by guessing, but it is not very likely (it only happened in about
1.2\% of our simulations). 
So \emph{one of two things must be true}:
\begin{itemize}
\item The lady got unusually ``lucky", or 
\item The lady is not just guessing.
\end{itemize}

Although Fisher did not say how the experiment came out, others have reported
that the lady correctly identified all 10 cups!
\cite{salsburg}


\subsubsection{A different design}

Suppose instead that we prepare five cups each way (and that the woman tasting
knows this).  We give her five cards labeled ``milk first'', and she must place 
them next to the cups that had the milked poured first.  How does this 
design change things?

\begin{knitrout}
\definecolor{shadecolor}{rgb}{.97, .97, .97}{\color{fgcolor}\begin{kframe}
\begin{flushleft}
\ttfamily\noindent
\hlsymbol{results}{\ }\hlassignement{\usebox{\hlnormalsizeboxlessthan}-}{\ }\hlfunctioncall{do}\hlkeyword{(}\hlnumber{10000}\hlkeyword{)}{\ }\hlkeyword{*}{\ }\hlfunctioncall{table}\hlkeyword{(}\hlfunctioncall{sample}\hlkeyword{(}\hlfunctioncall{c}\hlkeyword{(}\hlstring{"{}M"{}}\hlkeyword{,}{\ }\hlstring{"{}M"{}}\hlkeyword{,}{\ }\hlstring{"{}M"{}}\hlkeyword{,}{\ }\hlstring{"{}M"{}}\hlkeyword{,}{\ }\hlstring{"{}M"{}}\hlkeyword{,}{\ }\hlstring{"{}T"{}}\hlkeyword{,}\hspace*{\fill}\\
\hlstd{}{\ }{\ }{\ }{\ }\hlstring{"{}T"{}}\hlkeyword{,}{\ }\hlstring{"{}T"{}}\hlkeyword{,}{\ }\hlstring{"{}T"{}}\hlkeyword{,}{\ }\hlstring{"{}T"{}}\hlkeyword{)}\hlkeyword{,}{\ }\hlnumber{5}\hlkeyword{)}\hlkeyword{)}\hspace*{\fill}\\
\hlstd{}\hlfunctioncall{table}\hlkeyword{(}\hlsymbol{results}\hlkeyword{\usebox{\hlnormalsizeboxdollar}}\hlsymbol{M}\hlkeyword{)}\hlkeyword{/}\hlnumber{10000}\mbox{}
\normalfont
\end{flushleft}
\begin{verbatim}

     1      2      3      4      5 
0.1005 0.3892 0.3982 0.1037 0.0049 
\end{verbatim}
\end{kframe}}
\end{knitrout}





